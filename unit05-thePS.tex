%  For slides only
%% Usage:
%% $ cat slidesonly.tex unitXX-foo.tex | pdflatex
%% $ pdfjam -o unitXX-slides{a/b/...}.pdf texput.pdf '1,{1stpage}-{lastpage}'

\documentclass{beamer} 
\newcommand{\tcw}{\textcolor{black}}
\newcommand{\mynoteonly}{}
\newcommand{\nottheirhandout}{handout:0}

% % For handout
%%% Usage:
%% $ cat handout.tex unitXX-foo.tex | pdflatex
%% $ pdfjam -o unitXX-handout{a/b/...}.pdf texput.pdf '{1stpage}-{lastpage}'
\documentclass[handout]{beamer}
\usepackage{pgfpages}
\pgfpagesuselayout{4 on 1}[letterpaper,border shrink=5mm]%\nofiles

\mode<handout>{\setbeamercolor{background canvas}{bg=black!5}}
\newcommand{\tcw}{\textcolor{structure.bg}}
\newcommand{\mynoteonly}{| handout:0}
\newcommand{\nottheirhandout}{handout:0}

%For handout + mynotes
%%% Usage:
%% $ cat handout+mynotes.tex unitXX-foo.tex | pdflatex
%% $ pdfjam --nup 1x4 -o unitXX-foo-wmn{a/b/...}.pdf texput.pdf '1,{1stpage}-{lastpage}'
\documentclass[handout]{beamer}
\usepackage{pgfpages}
%%\pgfpagesuselayout{2 on 1}[letterpaper,border shrink=5mm] 
\setbeameroption{show notes on second screen=left}
\newcommand{\tcw}{\textcolor{black}}
\newcommand{\mynoteonly}{}
\newcommand{\nottheirhandout}{}

\newcommand{\igrphx}[2][width=\linewidth]{\includegraphics[#1]{images/#2}}

\renewcommand{\strut}{\rule{0pt}{3ex}}

% Frame note commands, with time budget updating
\newcounter{timeTotal}
\newcommand{\printUpdateTimeTotal}[1]{ \addtocounter{timeTotal}{#1}
  \textrm{(#1 min budgeted)} \hfill \textbf{Finish by
    \arabic{timeTotal}\ min from start}\\[1ex]}
\newcommand{\tnote}[3][10]{#2 \note{#2: #3 \\[.5ex]} \printUpdateTimeTotal{#1}}
\newcommand{\itnote}[2][10]{\note{ \begin{itemize} #2 \end{itemize}
\vfill \noindent  \printUpdateTimeTotal{#1} }}
\newcommand{\ennote}[2][10]{\note{ \begin{enumerate} #2 \end{enumerate}
\vfill \noindent  \printUpdateTimeTotal{#1} }}
\newcommand{\Note}[2][10]{\note{ #2 \mbox{ }\\ \vfill \noindent  \printUpdateTimeTotal{#1} }}

\newenvironment{Column}[1][.5\linewidth]{\begin{column}{#1}}{\end{column}}

\mode<handout>{\beamertemplatesolidbackgroundcolor{black!5}} 
\mode<article>{\usepackage{fullpage}}
\mode<presentation>
{
  \usetheme{Boadilla}
  % or ...

  \setbeamercovered{transparent}
  % or whatever (possibly just delete it)
}
\usepackage{url}
\usepackage{ulem}

\usepackage[english]{babel}
% or whatever

\usepackage[latin1]{inputenc}
% or whatever

\usepackage{times}
\usepackage[T1]{fontenc}
% Or whatever. Note that the encoding and the font should match. If T1
% does not look nice, try deleting the line with the fontenc.

\AtBeginSubsection[]
{
  \begin{frame}<beamer>
    \frametitle{Outline}
    \tableofcontents[subsectionstyle=show/shaded/hide]
  \end{frame}
}

\AtBeginSection[]
{
  \begin{frame}<beamer>
    \frametitle{Outline}
    \tableofcontents[hideothersubsections,sectionstyle=show/shaded]
  \end{frame}
}


% If you wish to uncover everything in a step-wise fashion, uncomment
% the following command: 

%\beamerdefaultoverlayspecification{<+->}

%USEFUL CODE TEMPLATES:
%\begin{itemize}[<+-| alert@+>]


% \begin{frame}[fragile]


% \begin{columns}
% \column{.4\textwidth}  
%   \begin{itemize}
%   \item<1-| alert@1> Why sample twice?
%   \item<2-| alert@2> Illustrative example.
%   \item<3-| alert@3> PSU's and SSU's.
%   \item<4-| alert@4> Stratification in combination with two-stage cluster sampling.
%   \item<5-| alert@5> Detailed example.
%   \end{itemize}
% \column{.6\textwidth}
% \only<2| handout:0>{
% \includegraphics[width=\textwidth]{nursingHomes}}
% \only<5| handout:1>{
% \includegraphics[width=\textwidth]{hosp_cover}}
% \end{columns}

% FOR INCLUDING R CODE. 
%\usepackage{listings}
%\lstset{language=R}
% \begin{frame}[fragile]
% \frametitle{Some code}
% \begin{lstlisting}
% > plot(myobj)
% > rm(myobj)
% \end{lstlisting}  
% \end{frame}

%\addtocounter{framenumber}{-1}


\usepackage{amsmath,amsthm}
\usepackage{wasysym,pifont}
\usepackage{Sweave}
\usepackage{ulem}
\usepackage{textcomp}
\usepackage{versions}
%% amsthm-type theorem environment specifications -- 
%% see amsthdoc.pdf in amscls documentation
\theoremstyle{plain}
\newtheorem{prop}{Proposition}[section]
\newtheorem{lem}[prop]{Lemma}

\newtheorem*{thm}{Proposition}
\newtheorem*{cor}{Corollary}

\theoremstyle{definition}
\newtheorem{defn}{Definition}[section]

\newcommand{\Pdistsym}{P}
\newcommand{\Pdistsymn}{P_n}
\newcommand{\Qdistsym}{Q}
\newcommand{\Qdistsymn}{Q_n}
\newcommand{\Qdistsymni}{Q_{n_i}}
\newcommand{\Qdistsymt}{Q[t]}
\newcommand{\dQdP}{\ensuremath{\frac{dQ}{dP}}}
\newcommand{\dQdPn}{\ensuremath{\frac{dQ_{n}}{dP_{n}}}}
\newcommand{\EE}{\ensuremath{\mathbf{E}}}
\newcommand{\EEp}{\ensuremath{\mathbf{E}_{P}}}
\newcommand{\EEpn}{\ensuremath{\mathbf{E}_{P_{n}}}}
\newcommand{\EEq}{\ensuremath{\mathbf{E}_{Q}}}
\newcommand{\EEqn}{\ensuremath{\mathbf{E}_{Q_{n}}}}
\newcommand{\EEqni}{\ensuremath{\mathbf{E}_{Q_{n[i]}}}}
\newcommand{\EEqt}{\ensuremath{\mathbf{E}_{Q[t]}}}
\newcommand{\PP}{\ensuremath{\mathbf{Pr}}}
\newcommand{\PPp}{\ensuremath{\mathbf{Pr}_{P}}}
\newcommand{\PPpn}{\ensuremath{\mathbf{Pr}_{P_{n}}}}
\newcommand{\PPq}{\ensuremath{\mathbf{Pr}_{Q}}}
\newcommand{\PPqn}{\ensuremath{\mathbf{Pr}_{Q_{n}}}}
\newcommand{\PPqt}{\ensuremath{\mathbf{Pr}_{Q[t]}}}
\newcommand{\var}{\ensuremath{\mathbf{V}}}
\newcommand{\varp}{\ensuremath{\mathbf{V}_{P}}}
\newcommand{\varpn}{\ensuremath{\mathbf{V}_{P_{n}}}}
\newcommand{\varq}{\ensuremath{\mathbf{V}_{Q}}}
\newcommand{\cov}{\ensuremath{\mathbf{Cov}}}
\newcommand{\covp}{\ensuremath{\mathbf{Cov}_{P}}}
\newcommand{\covpn}{\ensuremath{\mathbf{Cov}_{P_{n}}}}
\newcommand{\covq}{\ensuremath{\mathbf{Cov}_{Q}}}

\newcommand{\hatvar}{\ensuremath{\widehat{\mathrm{Var}}}}
\newcommand{\hatcov}{\ensuremath{\widehat{\mathrm{Cov}}}}

\newcommand{\sehat}{\ensuremath{\widehat{\mathrm{se}}}}

\newcommand{\combdiff}[1]{\ensuremath{\Delta_{{z}}[#1]}}
\newcommand{\Combdiff}[1]{\ensuremath{\Delta_{{Z}}[#1]}}

\newcommand{\psvec}{\ensuremath{\varphi}}
\newcommand{\psvecgc}{\ensuremath{\tilde{\varphi}}}


\newcommand{\atob}[2]{\ensuremath{#1\!\! :\!\! #2}}
\newcommand{\stratA}{\ensuremath{\mathbf{S}}}
\newcommand{\stratAnumstrat}{\ensuremath{S}}
\newcommand{\sAsi}{\ensuremath{s}}

\newcommand{\permsd}{\ensuremath{\sigma_{\Pdistsym}}}
% \newcommand{\dz}[1]{\ensuremath{d_{z}[{#1}]}}
\newcommand{\dZ}[1]{\ensuremath{d_{Z}[{#1}]}}
\newcommand{\tz}[1]{\ensuremath{t_{{z}}[#1]}}
\newcommand{\tZ}[1]{\ensuremath{t_{{Z}}[#1]}}


\newlength{\tabcolsepadj}
\setlength{\tabcolsepadj}{1.3mm}

%%% NEWBLOCK UNDEFINED BUG
\def\newblock{\hskip .11em plus .33em minus .07em}

%%% tightlist undefined control sequence bug 
%%% http://tex.stackexchange.com/questions/257418/error-tightlist-converting-md-file-into-pdf-using-pandoc
\providecommand{\tightlist}{%
  \setlength{\itemsep}{0pt}\setlength{\parskip}{0pt}}
\newcommand{\igrphx}[2][width=\linewidth]{\includegraphics[#1]{images/#2}}

\renewcommand{\strut}{\rule{0pt}{3ex}}

\newenvironment{Column}[1][.5\linewidth]{\begin{column}{#1}}{\end{column}}

\usetikzlibrary{arrows} % also see \tikzstyle spec below after \begin{doc}

\newcommand{\mlpnode}[1]{\tikz[baseline=-.5ex] \coordinate (#1) {};}
%\newcommand{\mlpnode}[1]{\raisebox{.5ex}{\pnode{#1}}}


\usepackage{xspace}

\includeversion{pedantic} % \excludeversion

\tikzstyle{every picture}+=[remember picture]


\date{ICPSR Session 2 (\today)}
\usepackage{natbib}
\newcounter{saveenumi}
\title{Unit 5: The propensity score}
% \author, date moved to beamer-preamble-*-all.tex

\begin{document}







  % \alert{Readings for this Unit:}\\
  % \begin{itemize}
  % \item Rosenbaum (2002), ``Covariance adjustment in randomized experiments and observational  studies, '' \S 2.3 ({\em Statistical Science}, 17/3:286--327).
  % \end{itemize}

\input{announcement-of-the-day}  % announcement-of-the-day.tex not
                                % part of repo



\section[Extrapolation \& overlap]{Pitfalls of multiple covariates:
  Extrapolation \& overlap}


\begin{frame} \frametitle{Contrasting two groups after adjustment for a covariate}
\framesubtitle{The case of a single continuous
  covariate\footnote{Trochim, ``Nonequivalent groups design,''
    \url{socialresearchmethods.net}; Fig.~1 of Rubin (1977, \textit{J. Educ. Statist.} \textbf{2}/1 1--26.).}}

\begin{center}
  \only<1\mynoteonly>{\igrphx[height=.6\textheight]{pretest-comp-group-design}}
%  \only<2>{\igrphx[height=.8\textheight]{rubin1977fig1}}
\end{center}

\end{frame}
\note[itemize]{
\item {}\textit{[Explain picture]}
\item Even if we knew these curves, we'd have to average both curves
  over a common standard distribution in order to estimate a causal effect.
  \item Even if we knew these curves, there'd be ETT, ETC, \ldots
    \item Problem: we don't know the 2 curves.  Further, if we try to
      estimate from data, might have to extrapolate.  In that case
      have to be very careful!
}

\begin{frame}[label=contrast2grFr]
  \frametitle{Contrasting two groups after adjustment for a covariate}
\framesubtitle{Preventing extrapolation with two covariates}

\begin{center}
  \igrphx[angle=270,width=\linewidth]{mvextrapsketch}
\end{center}

\end{frame}

\note[itemize]{
  \item W/ multiple variables, extrapolation can't necessarily be identified by looking for extrapolation on the variables individually.  
  \item The problem gets worse as the number of variables increases.
  }

\begin{frame}
  \frametitle{Contrasting two groups after adjustment for a covariate}
\framesubtitle{Preventing extrapolation with \textit{many}
  covariates\citep[\textit{cf.} ][\textit{Ann. Intern. Med.}]{rubin:1997}}

  \begin{center}
    \igrphx[height=.9\textheight]{psboxplot}
  \end{center}

\end{frame}

\note[itemize]{
\item The are ``propensity scores'' -- to be defined.
\item Propensity scores are immediate children of ``discriminant scores.'' As name suggests, their purpose is to discriminate between the groups.
\item The scores fold together a potentially large number of variables.  So plots like this inform us about extent of extrapolation problem.
\item Jargon: ``overlap''; ``comon support''.
\item Role of assumptions in comparisons outside region of overlap.  Minimizing assumptions often requires restricting the comparison, ie. ``throwing away data.''
}
\section{Balance in controlled and natural experiments}
%\mbox{}
\begin{frame}
  \frametitle{The Neyman-Rubin Model for (simple) experiments}
\vfill
  This is what randomization ensures:
$$ (\alert<4>{Y_t, Y_c},\alert<2>{X}) \perp \alert<3>{Z} $$
\vfill

\uncover<5->{I.e., each of $X$, $Y_{c}$ and $Y_{t}$ is balanced} \uncover<6->{(in expectation; modulo sampling variability).}

\end{frame}
\note[itemize]{
  \item In controlled experiments, random assignment warrants this.
  \item In natural experiments, justified otherwise, or an article of faith.
  \item In an experiment, the $x$es aren't necessary for inference.
  \item However, the part with the $x$es has testable consequences.}

 \begin{frame}%<1>[label=covbalexptsFr]
  \frametitle{Covariate balance in experiments}
%\only<2>{\framesubtitle{The importance of comparing it to balance in other experiments}}

\begin{columns}
  \begin{column}{.4\linewidth}
    \begin{itemize}
    \item<1-> \citet{arceneaux:2005}
    \item<1-> Kansas City, November 2003
    \item<1-> Completely randomized design: 14 precincts $\rightarrow$ Tx; 14 $\rightarrow $ Control.

    \item<1-> Substantively large baseline differences
%    \item<2-> Differences not large compared to other assignments from same design.
%    \item<2-> $\PP(\chi^{2} > x) = .94$ \citep{hansen:bowers:2008}.
    \end{itemize}
  \end{column}
  
\begin{column}{.6\linewidth}
\only<1>{\igrphx{KC-baseline}}
\only<2>{\igrphx{KC-bal+SDs}}
\end{column}
\end{columns}

\end{frame}

%  \begin{frame}
%\frametitle{Balance in randomized experiments}
% \begin{center}
%   \igrphx[width=4in]{balanceAtBaseline}
% \end{center}

% \enlargethispage*{1000pt}
% \only<1>{\footnotesize Table 1 of Bray et al. (2002), ``A 9-mo
%   randomized clinical trial comparing fat-substituted and fat-reduced
%   diets in healthy obese men: the Ole Study,''
%   \emph{Am. J. Clin. Nutr.} \textbf{76}/5, 928--934.}
% \only<2>{Balance due to randomized assignment: not perfect, but good enough.}
%\end{frame}

\begin{frame}[fragile] 
  \frametitle{1971 UC Berkeley Admissions}

\bigskip

Proportions of men and women \textit{applying} to the 6 departments.\\
% latex table generated in R 3.2.3 by xtable 1.8-2 package
% Fri Jul 29 14:16:35 2016
\begin{tabular}{rrrrrrrr}
  \hline
 & A & B & C & D & E & F & Overall \\ 
  \hline
Male & 31 & 21 & 12 & 15 & 7 & 14 & 100 \\ 
  Female & 6 & 1 & 32 & 20 & 21 & 19 & 100 \\ 
  diff. & 25 & 19 & -20 & -5 & -14 & -5 & 0 \\ 
   \hline
\end{tabular}\\
\bigskip

Define dummy variables \texttt{DeptA}, \ldots, \texttt{DeptF}. Comparison is \textit{balanced} for \texttt{DeptA} if M, W means of \texttt{DeptA} are similar.
% \[ \mathtt{a}_{i}=\left\{
%   \begin{array}{lr}
%     1, & $i$\, \mathrm{applied\ to\ dept\ A}\\
%     0, & \mathrm{otherwise}.
%   \end{array}
% \right.\]
% The comparison is \textit{balanced} on \texttt{a} if $\bar{\mathtt{a}}_{m}
% \approx \bar{\mathtt{a}}_{f}$, balanced on \texttt{b} if $\bar{\mathtt{b}}_{m}
% \approx \bar{\mathtt{b}}_{f}$, etc.  \pause

{\small
\begin{Schunk}
\begin{Sinput}
> xBalance(Female ~ Dept, data=UCBA, 
+          report=c("adj.means", "adj.mean.diffs"))
\end{Sinput}
\begin{Soutput}
      strata  unstrat                           
      stat   Female=0 Female=1 adj.diff         
vars                                            
DeptA        3.1e-01  5.9e-02  -2.5e-01 ***     
DeptB        2.1e-01  1.4e-02  -1.9e-01 ***     
DeptC        1.2e-01  3.2e-01  2.0e-01  ***     
DeptD        1.5e-01  2.0e-01  4.9e-02  ***     
DeptE        7.1e-02  2.1e-01  1.4e-01  ***     
DeptF        1.4e-01  1.9e-01  4.7e-02  ***     
\end{Soutput}
\end{Schunk}
}

\vfill
\end{frame}


\note[itemize]{
\item Explain lack of balance on dept.
% \begin{pedantic}
% \item  ``Loose Definition''
% \item Lack of balance on one or more covariates is a precondition for Simpson.
% \end{pedantic}
\item Balance in a stratified obs study can be compared to balance in a block-randomized experiment, as lower panel does.
\item Significance stars at left report 6 Pearson chi-square tests.  In an experiment, would general three stars less than .1\% of random assignments.
}



\begin{frame}
  \frametitle{The Neyman-Rubin Model for stratified experiments}

  This is what randomization ensures in a \textit{simple} experiment:
$$ ({Y_t, Y_c},{X}) \perp {Z} .$$

In a \textit{stratified} experiment, this becomes:
$$ ({Y_t, Y_c},{X}) \perp {Z} | \alert<2>{S} .$$
In words, there are independent simple experiments in each stratum.  

\uncover<3->{Methods of outcome analysis, as well as methods of inspecting covariate balance, have to be updated to account for the additional structure.}


\end{frame}

\begin{frame}[fragile,label=SpafFr] 
%  \frametitle{Simpson's paradox: A fix}
\frametitle{Combining contrasts across strata}

How to compare male and female acceptance rates? 
% latex table generated in R 3.2.3 by xtable 1.8-2 package
% Fri Jul 29 14:16:35 2016
\begin{tabular}{rrrrrrrrr}
  \hline
 & A & B & C & D & E & F & Unweighted & n.apps.weighted \\ 
  \hline
Male & 62 & 63 & 37 & 33 & 28 & 6 & 45 & 39 \\ 
  Female & 82 & 68 & 34 & 35 & 24 & 7 & 30 & 43 \\ 
  diff & -20 & -5 & 3 & -2 & 4 & -1 & 14 & -4 \\ 
  \hline n.apps & 21 & 13 & 20 & 17 & 13 & 16 &  &  \\ 
   \hline
\end{tabular}
\pause

One way that
doesn't acknowledges departmental differences, and two ways that do. 
\begin{enumerate}
 \item Compute Male and Female rates; compare. 
\item<2-> \label{item:4}Take a weighted average of \texttt{diff}, weighting in
  proportion to, say, \texttt{n.apps}.
\item<3-> \label{item:5}Standardize male and female acceptance rates, using \texttt{n.apps} to define the standard population; subtract standardized female rate from standardized male rate.
\end{enumerate}

\uncover<4->{
Both (\ref{item:4}) and (\ref{item:5}) give the same answer.
}
\end{frame}

\itnote[10]{
% \begin{pedantic}
% \item {To pool the by-department comparisions, natural to take weighted average of differences.  (How does this differ from just averaging men and then women, without regard to dept.?)}
% \end{pedantic}

\item {Weighting the differences by the number of applicants to each department gives -0.043.\\
}
\item {\textbf{A good time to stop and have them} verify my
    equivalence claim using the Simpson's paradox examples that they
    created earlier.} %+5 to time estimate for this frame
}
\begin{frame}[fragile]
  \frametitle{Combining across strata: varying the recipe}

Different standards give different assessments of the male advantage, all of them adjusted.  Some possible standard populations:
\begin{itemize}
\item  \alert<2| handout:0>{all applicants, categorized by dept};
\item  \alert<3| handout:0>{Depts' \textit{harmonic means} of \# M, \# F applicants };
\item  \alert<4| handout:0>{all M applicants, categorized by dept };
\item  \alert<5| handout:0>{all F applicants, categorized by dept
    (compare to ``effect of treatment on the treated'' [ETT])}.
\end{itemize}


% latex table generated in R 3.2.3 by xtable 1.8-2 package
% Fri Jul 29 14:16:35 2016
\begin{tabular}{rrrrrrrrr}
  \hline
 & A & B & C & D & E & F & unweighted & weighted \\ 
  \hline
diff & -20 & -5 & 3 & -2 & 4 & -1 & 14 &  \\ 
  \alert<2| handout:0>{n.apps} & 21 & 13 & 20 & 17 & 13 & 16 &  & -4 \\ 
  \alert<3| handout:0>{h.wts} & 11 & 3 & 25 & 24 & 15 & 21 &  & -7 \\ 
  \alert<4| handout:0>{n.m} & 31 & 21 & 12 & 15 & 7 & 14 &  & -0 \\ 
  \alert<5| handout:0>{n.f} & 6 & 1 & 32 & 20 & 21 & 19 &  & -2 \\ 
   \hline
\end{tabular}


\end{frame}
\itnote{
\item \texttt{n.apps} simplest to describe, but it gives similar weights
  to Depts A and C on the basis of their similar sizes, despite the
  fact that A had was v small in terms of F applicants, making its
  total size somewhat deceptive as an indicator of what it can
  contribute to this comparison.
\item harmonic takes into account the sizes of both groups in each
  stratum.  (\textbf{Calculate for own Simpson's example as a HW?} )
\item The \# of men applicants is difficult to motivate, but \# women
  is v natural, as women are the focus of the study.
}

\begin{frame}[fragile]
  \frametitle{Using the \textsc{anova} to combine across strata}

 Using OLS/\textsc{anova} to address the problem gives
\begin{Schunk}
\begin{Sinput}
> lm(Admit~Gender+Dept, data=UCBA)
\end{Sinput}
\end{Schunk}
\begin{Schunk}
\begin{Soutput}
             Estimate Std. Error t value Pr(>|t|)
(Intercept)     0.642      0.015   43.88     0.00
GenderFemale    0.018      0.015    1.20     0.23
DeptB          -0.010      0.023   -0.44     0.66
\end{Soutput}
\end{Schunk}
$\vdots$

 Recall that using harmonic means of the departments' numbers of men and women applicants to standardize led to an adjusted difference of $-1.84\% $.
  
\end{frame}
\itnote{
\item So the ANOVA t-test uses harmonic weighting.  
\item So does the CMH test, which is designed for binary data. 
\item ``Precision weighting.''
}

%\begin{pedantic}
\begin{frame}[fragile]
  \frametitle{(Aside: \texttt{xBalance()} and \texttt{lm()} both use precision weighting)}

\begin{Schunk}
\begin{Sinput}
> xBalance(Female ~ Dept+Admit, strata=list(Department=~Dept),
+          data=UCBA, report=c("adj.means","adj.mean.diffs"))
\end{Sinput}
\begin{Soutput}
          strata Department                           
          stat     Female=0 Female=1 adj.diff         
vars                                                  
DeptA              1.1e-01  1.1e-01  -1.4e-15         
DeptB              2.9e-02  2.9e-02  9.3e-17          
DeptC              2.5e-01  2.5e-01  -4.4e-15         
DeptD              2.4e-01  2.4e-01  2.4e-15          
DeptE              1.5e-01  1.5e-01  7.9e-16          
DeptF              2.1e-01  2.1e-01  -3.4e-17         
AdmitTRUE          3.2e-01  3.3e-01  1.8e-02          
\end{Soutput}
\end{Schunk}
\end{frame}
\note[itemize]{
\item Weighting produces both \texttt{Female=0} and \texttt{Female=1} columns
\item Although maybe not immediately clear, the \texttt{adj.diff}'s for \texttt{DeptX} variables are effectively 0
\item As they should be
\item The adj.diff for \texttt{AdmitTRUE} is .018 --- \textit{not} 0; also the same as we got from the ANOVA.
}
%\end{pedantic}

\begin{frame}%<1>[label=covbalexptsFr]
  \frametitle{Covariate balance in experiments}
\only<2>{\framesubtitle{The importance of comparing it to balance in other experiments}}

\begin{columns}
  \begin{column}{.4\linewidth}
    \begin{itemize}
    \item<1-> \citet{arceneaux:2005}
    \item<1-> Kansas City, November 2003
    \item<1-> Completely randomized design: 14 precincts $\rightarrow$ Tx; 14 $\rightarrow $ Control.

    \item<1-> Substantively large baseline differences
    \item<1-| alert@+> Differences not large compared to other assignments from same design.
    \item<2-> $\PP(\chi^{2} > x) = .94$ \citep{hansen:bowers:2008}.
    \end{itemize}
  \end{column}
  
\begin{column}{.6\linewidth}
\only<1| handout:0>{\igrphx{KC-baseline}}
\only<2>{\igrphx{KC-bal+SDs}}
\end{column}
\end{columns}

\end{frame}

\Note{
There's also a natural way of combining imbalances on different
measures into a single chi-square measure.
}


\section{Inducing balance with the propensity score}
%\frame{\tableofcontents[]}


\begin{frame}
  \frametitle{A limitation}

  Propensity-score methods are made for \textit{prospective} designs;
  application to retrospective (case-control) studies is limited.
\end{frame}

%\mbox{}
\begin{frame}<1-3>[label=thePSfr]
  \frametitle{The propensity score}

Given covariates $\mathbf{X} (=(X_1, \ldots, X_k))$, and a
treatment variable $Z$, $Z(u) \in \{0, 1\}$,  $\PP (Z \vert \mathbf{X})$ is known as the (true) \alert<1>{propensity score} (PS).  
$$ \phi( \mathbf{X} ) \equiv \log\left( \PP (Z=1 \vert \mathbf{X})/\PP (Z=0 \vert \mathbf{X}) \right)$$
is also known as the PS.  In practice, one works
with an estimated PS, $\hat{\PP} (Z \vert \mathbf{X})$ or
$\hat{\phi}(\mathbf{X})$.

Theoretically, propensity-score strata or matched sets both
\begin{enumerate}
 \item<2-| alert@+> reduce extrapolation; and
\item<3-| alert@+> balance each of $X_1, \ldots, X_k$.
\end{enumerate}
\uncover<4->{They do this by making the comparison more
  ``experiment-like'', at least in terms of $X_1, \ldots, X_k$.}

\uncover<5->{Theory also tells us that in the absence of hidden bias, such a stratification}
\begin{enumerate}
  \addtocounter{enumi}{2}
\item<5-| alert@+> supports unbiased estimation of treatment effects.
\end{enumerate}

\end{frame}



\againframe<4-5>{thePSfr}
%\mbox{}
\begin{frame}
  \frametitle{Propensity scoring in practice}

\enlargethispage*{1000pt}

\begin{columns}
  \column{.65\linewidth}
\begin{enumerate}[<+-| alert@+>]
\item Fitted propensity scores help identify extrapolation.
\item In practice, stratification on $\hat{\phi}(\mathbf{X})$
helps balance each of $X_1, \ldots, X_k$.
% \item Adjustment based on the
% propensity score is arguably more transparent and more stable
% that adjustment based on multiple regression.
\item There are \emph{lots of cases} in which adjustment with the propensity
score fails to generate estimates that square with those of
randomized studies. \setcounter{saveenumi}{\value{enumi}}
\end{enumerate}
\column{.35\linewidth} \igrphx{psboxplot}
\end{columns}

\begin{enumerate}[<+-| alert@+>] \setcounter{enumi}{\value{saveenumi}}
\item There are various reasons for this, starting with: lots of observational studies that
  don't measure quite enough $x$es.
\item (Propensity scores address bias on measured variables, not
  unmeasured ones.)  \textit{hidden bias}.
\end{enumerate}

\end{frame}

\begin{frame}
\frametitle{Bias removal with an ``estimated'' ``propensity
score''}

In the UCB Admissions data set, applicants to departments A and B
appear to have roughly similar ``feminine propensities,'' as do
applicants to departments C and E, or to D and F.

%\only<1-2>
{
 \begin{tabular}[b]{|l|rrrrrr|} \hline
& \multicolumn{6}{|c|}{Department} \\
&   A &   B &   C &   D &   E &   F \\ \hline
\% Female & .12 & .04 & .65 & .47 & .67 & .48 \\ \hline
\end{tabular}
 \pause
 \begin{tabular}[b]{rrr|} \hline
  \multicolumn{3}{c|}{Female App. \%}\\
  Low & Med. & High \\ \hline
  .09 & .66 & .48 \\ \hline
\end{tabular}
}

How does this ``propensity'' stratification affect balance?
% \begin{onlyenv}<3>%{
%   An analysis that groups applicants to departments with similar
% ``feminine propensities'' also removes the bias in the original
% data.
% {\small
% \begin{center}
% 1971 UCB Admissions \citep{Bickel:etal:1975}
% \begin{tabular}{|ll|r|rrr|} \hline
%          &       & Ignoring & \multicolumn{3}{|c|}{PS Stratum} \\
% Admit    &Gender &department&Low& Med & High \\ \hline
% Admitted &Male   & 1200    &860 & 170 & 160 \\
%          &Female &  560    &110 & 300 & 160 \\
% Rejected &Male   & 1500    &520 & 340 & 630 \\
%          &Female & 1300    & 30 & 690 & 560 \\ \hline
% \multicolumn{2}{|c|}{$\chi_1^2$-statistic} & $M^2=91.6$ &\multicolumn{3}{|c|}{$X^2=1.2$} \\
% \multicolumn{2}{|c|}{$p$-value } & $p \leq 2.2\cdot 10^{-16}$
% &\multicolumn{3}{|c|}{$p=.28$} \\ \hline
% \end{tabular}
% \end{center}
% }
% %}
% \end{onlyenv}
\end{frame}

\begin{frame}[fragile]
  \frametitle{Bias removal with an ``estimated'' ``propensity score''}
\begin{Schunk}
\begin{Sinput}
> UCBA$PS.hat <- predict( glm(Female~Dept, binomial, data=UCBA) )
> table(round(UCBA$PS.hat, 2))
\end{Sinput}
\begin{Soutput}
-3.11 -2.03 -0.11 -0.09   0.6  0.72 
  585   933   792   714   918   584 
\end{Soutput}
\begin{Sinput}
> UCBA$PSS <- cut(UCBA$PS.hat, breaks=c(-4,-2,0,2))
\end{Sinput}
\end{Schunk}
% \begin{Schunk}
% \begin{Sinput}
% > UCBA$PSS <- UCBA$Dept ; levels(UCBA$PSS)
% \end{Sinput}
% \begin{Soutput}
% [1] "A" "B" "C" "D" "E" "F"
% \end{Soutput}
% \begin{Sinput}
% > levels(UCBA$PSS) <-  scan(what="character", nlines=1)
% 1: Low Low Hi Med Hi Med
% Read 6 items
% \end{Sinput}
% \end{Schunk}

% \begin{verbatim}
% xBalance(Female ~ Dept, strata=data.frame(none='-',
% +   UCBA[c("PSS", "Dept")]), data=UCBA, report=<...>)
% \end{verbatim}
\begin{Schunk}
\begin{Sinput}
> xBalance(Female ~ Dept, data=UCBA, report=c("adj.mean.diffs"), 
+          strata=data.frame(`none`="-", UCBA["PSS"])     )
\end{Sinput}
\begin{Soutput}
      strata     none               PSS         
      stat   adj.diff          adj.diff         
vars                                            
DeptA        -2.5e-01 ***      3.1e-02  ***     
DeptB        -1.9e-01 ***      -3.1e-02 ***     
DeptC        2.0e-01  ***      -1.2e-02         
DeptD        4.9e-02  ***      -1.8e-03         
DeptE        1.4e-01  ***      1.2e-02          
DeptF        4.7e-02  ***      1.8e-03          
\end{Soutput}
\end{Schunk}

\end{frame}
\note[itemize]{
\item Propensity subclassification isn't as good as exact matching on
  Dept, but much better than nothing.
\item Propensity works better for C--F than for A and B.
\item Note that you don't have to have the same value on the
  underlying covariate to have the same propensity score.  We balanced
  C--F, but we didn't match exactly on them.
\item Dimension reduction: turned a 5-dimensional covariate into a
3-dimensional one.

}
%\mbox{}

\begin{frame}[label=FemPptyFr]
\frametitle{Bias removal with an ``estimated'' ``propensity
score''}

In the UCB Admissions data set, applicants to departments A and B
appear to have roughly similar ``feminine propensities,'' as do
applicants to departments C and E, or to D and F.

%\only<1-2>
{
 \begin{tabular}[b]{|l|rrrrrr|} \hline
& \multicolumn{6}{|c|}{Department} \\
&   A &   B &   C &   D &   E &   F \\ \hline
\% Female & .12 & .04 & .65 & .47 & .67 & .48 \\ \hline
\end{tabular}
 \pause
 \begin{tabular}[b]{rrr|} \hline
  \multicolumn{3}{c|}{Female App. \%}\\
  Low & Med. & High \\ \hline
  .09 & .66 & .48 \\ \hline
\end{tabular}
}

%\begin{onlyenv}<3>%{
  An analysis that groups applicants to departments with similar
``feminine propensities'' also removes the bias in the original
data.
{\small
\begin{center}
1971 UCB Admissions \citep{Bickel:etal:1975}
\begin{tabular}{|ll|r|rrr|} \hline
         &       & Ignoring & \multicolumn{3}{|c|}{PS Stratum} \\
Admit    &Gender &department&Low& Med & High \\ \hline
Admitted &Male   & 1200    &860 & 170 & 160 \\
         &Female &  560    &110 & 300 & 160 \\
Rejected &Male   & 1500    &520 & 340 & 630 \\
         &Female & 1300    & 30 & 690 & 560 \\ \hline
\multicolumn{2}{|c|}{$\chi_1^2$-statistic} & $M^2=91.6$ &\multicolumn{3}{|c|}{$X^2=1.2$} \\
\multicolumn{2}{|c|}{$p$-value } & $p \leq 2.2\cdot 10^{-16}$
&\multicolumn{3}{|c|}{$p=.28$} \\ \hline
\end{tabular}
\end{center}
}
%}
%\end{onlyenv}
\end{frame}
\Note{
 Very much a toy example, but real ones work in much the same way.
}


\begin{frame}
  \frametitle{Illustrative example: propensity stratification in an experiment}

\begin{columns}
  \begin{column}{.5\linewidth}
    \begin{itemize}
    \item The imbalance creates a small conditional bias in $\bar{y}_{t} - \bar{y}_{c}$.
    \item<2-> PS stratification reduces covariate imbalance \citep{hill:etal:2000} \& thus reduces ``bias''
   \end{itemize}
  \end{column}
  \begin{column}{.5\linewidth}
    \only<1| handout:0>{\igrphx{KC-bal+SDs}}
    \only<2->{\igrphx{KC-balwithPS}}
  \end{column}
\end{columns}
\end{frame}

% \begin{frame} \enlargethispage*{400pt}
% \frametitle{The propensity score as an instrument of balance}
% \begin{center}
% \igrphx[height=7.8cm]{tomlovepic-lb-flat}
% \end{center}
% \end{frame}

%\begin{pedantic}
\begin{frame}
  \frametitle{Matching on a covariate versus matching to balance it}

  \begin{itemize}
\item   Distinct units $a$ and $b$ can be very
  different in terms of each of $X_1, \ldots, X_k$, so that
  $\mathbf{x}_a$ and $\mathbf{x}_b$ are very different, while at the same
  time $\hat{\phi}(\mathbf{x}_a) \approx \hat{\phi}(\mathbf{x}_b)$.
\item   Therefore matching on a propensity score involving $v$ (among other contributing covariates) will \emph{not} in general have the side effect of
  matching on $v$ itself.
\item It \textit{has} to be that way, because for most data sets it's mathematically infeasible to match closely on more than a couple of variables \citep{cochran:1972,abadie2006lsp}.
\end{itemize}

\note{
{\footnotesize

Benefits of matching on, as opposed to merely balancing, a variable:
  \begin{itemize}
\item
In some cases, an initial stratification is necessary to ensure that
comparison of outcomes is meaningful.
\item
If a covariate is strongly predictive of outcomes, matching or stratifying
upon it will tend to sharpen the eventual matched or stratified comparison.
\item If you want to study interactions of a variable with the treatment, have
to match on it, at least approximately.
  \end{itemize}

Cost of matching on as opposed to matching to balance:
\begin{itemize}
\item You can balance many more variables than you can match upon
\item Harder to find good matches on other variables; may force you to reduce sample size, either literally or figuratively.
\end{itemize}
}
}
\end{frame}
%\end{pedantic}





\begin{frame}[fragile,shrink]
  \frametitle{Stratification on a propensity score: some concerns}
\begin{Schunk}
\begin{Sinput}
> xBalance(Female ~ Dept, strata=data.frame(`none`="-", UCBA["PSS"]), 
+ data=UCBA, report=c("adj.mean.diffs"))
\end{Sinput}
\begin{Soutput}
      strata     none               PSS         
      stat   adj.diff          adj.diff         
vars                                            
DeptA        -2.5e-01 ***      3.1e-02  ***     
DeptB        -1.9e-01 ***      -3.1e-02 ***     
DeptC        2.0e-01  ***      -1.2e-02         
DeptD        4.9e-02  ***      -1.8e-03         
DeptE        1.4e-01  ***      1.2e-02          
DeptF        4.7e-02  ***      1.8e-03          
\end{Soutput}
\end{Schunk}
  \begin{itemize}[<+-| alert@+>]
  \item 
  Evidently, low propensity score stratum needs to be split.
\item In general, covariates won't align with stratum boundaries.
  \item Some software routines address this by ``testing'' for difference of means on the estimated propensity score, splitting a stratum when a difference is ``found.''
  \item Better remedies frame ``success'' of the stratification as a
    model, which is then tested using all the data at once.
\end{itemize}
\itnote{
  \item 
  Evidently, low propensity score stratum needs to be split.
\item In general, covariates won't line up neatly with stratum boundaries, so when they're out of balance it's difficult to see which strata may be too wide.
  \item Some software routines address this by ``testing'' for a difference of means on the estimated propensity score, splitting a stratum when a difference is ``found.''   But these tests ignore the estimated nature of the propensity score, and carry a multiplicity problem.
  \item 

}
\end{frame}


\nocite{hansen:klopfer:2006,rosenbaum:2002}


%\subsection{Balance checking in practice}

\begin{frame}
  \frametitle{Balance checking}
\begin{itemize}
\item \textit{Overt} vs. \textit{hidden} bias \citep{rosenbaum:2002}.
\item Balance checks address a form of overt bias.
\item Two ways of thinking about balance each suggests a method of assessing it:
  \begin{itemize}
  \item If covariate $x$ were the outcome variable, how biased would my treatment effect estimate of it be? (Hopefully not much.)
  \item How do average differences on $x$ compare with what we'd see under matched randomization?
  \end{itemize}
\end{itemize}
\end{frame}

\begin{frame}[fragile, shrink]
  \frametitle{If $x$ were the outcome, how biased would my treatment effect estimate be?}

  It's natural to look to effect sizes, between-group differences
  expressed as fractions of corresponding standard deviations, for a
  context-independent indication of severity of bias.
  
  
\begin{Schunk}
\begin{Sinput}
> xBalance(Female ~ Dept, strata=data.frame(`none`="-", UCBA["PSS"]), 
+ data=UCBA, report=c("adj.mean.diffs", "std.diffs"))
\end{Sinput}
\begin{Soutput}
      strata     none                        PSS                  
      stat   adj.diff std.diff          adj.diff std.diff         
vars                                                              
DeptA        -2.5e-01 -6.4e-01 ***      3.1e-02  8.1e-02  ***     
DeptB        -1.9e-01 -6.0e-01 ***      -3.1e-02 -9.8e-02 ***     
DeptC        2.0e-01  5.2e-01  ***      -1.2e-02 -3.0e-02         
DeptD        4.9e-02  1.3e-01  ***      -1.8e-03 -4.9e-03         
DeptE        1.4e-01  4.4e-01  ***      1.2e-02  3.5e-02          
DeptF        4.7e-02  1.3e-01  ***      1.8e-03  5.1e-03          
\end{Soutput}
\end{Schunk}

Standard reference points --- $.2s_{p}$, $.1s_{p}$ --- say this is
pretty good, despite significance stars.
\end{frame}

\begin{frame}[fragile]
\frametitle[Compare to matched randomization]{How do differences on $x$ compare with what we'd see under matched randomization?}

The model stating that propensity scores are the same within each
propensity stratum can be directly tested.

\begin{Schunk}
\begin{Sinput}
> xBalance(Female ~ Dept, strata=data.frame(`none`="-", UCBA["PSS"]), 
+ data=UCBA, report=c("chisquare.test"))
\end{Sinput}
\begin{Soutput}
---Overall Test---
     chisquare df  p.value
none      1068  5 1.1e-228
PSS         25  3  1.4e-05
---
Signif. codes:  0 
\end{Soutput}
\end{Schunk}

This gives a different picture.
\end{frame}

\begin{frame}{My \$.02}
 
  \begin{itemize}
  \item My advice \citep{hansen:bowers:2008,hansen:statmed:2008} is to
    do the chi-square test\footnote{\citet*{imaiKingStuart:2007}
      disagree, as does \citet*{austin2008critical}.  The Rosenbaum
      text (2009) and Stuart review \citeyearpar{stuart2010matching} have some discussion of the debate.}.
  \item If this goodness-of-fit test fails, we need a better model:
    ordinarily a finer stratification on the PS, or a closer match on it.
  \item Alternately, the chi-square test compares to a stratified
    experiment.  We know that covariance adjustment (regression) works
    pretty well in experiments.  If we've got the right $X$es,
    covariate adjustment should work in our non-experiment -- provide
    its balance on those $X$es is on par with that of a comparable experiment.
  \end{itemize}
  
  
  
\end{frame}

\section*{References}

%\mbox{}

\begin{frame}[allowframebreaks]
  \nocite{hansen:2004,hansen:klopfer:2006}
%%   {\scriptsize
%%     % \bibliographystyle{asa}
%%     % \bibliography{abbrev_long,causalinference,misc,computing,biomedicalapplications}
%%     \begin{thebibliography}{14}
%%       \newcommand{\enquote}[1]{``#1''}
%%       \expandafter\ifx\csname natexlab\endcsname\relax\def\natexlab#1{#1}\fi
%% 
%%       \bibitem[{Abadie and Imbens(2006)}]{abadie2006lsp}
%%       Abadie, A. and Imbens, G.~W. (2006), \enquote{{Large Sample Properties of
%%       Matching Estimators for Average Treatment Effects},} \textit{Econometrica},
%%       74, 235--267.
%% 
%%       \bibitem[{Arceneaux(2005)}]{arceneaux:2005}
%%       Arceneaux, K. (2005), \enquote{Using cluster randomized field experiments to
%%         study voting behavior,} \textit{Annals of the Americal Academy of Political
%%       and Social Science}, 601, 169--179.
%% 
%%       \bibitem[{Austin(2008)}]{austin2008critical}
%%       Austin, P.~C. (2008), \enquote{{A critical appraisal of propensity-score
%%       matching in the medical literature between 1996 and 2003},}
%%       \textit{Statistics in Medicine}, 27, 2037--2049.
%% 
%%       \bibitem[{Bickel et~al.(1975)Bickel, Hammel, and O'Connell}]{Bickel:etal:1975}
%%       Bickel, P.~J., Hammel, E.~A., and O'Connell, J.~W. (1975), \enquote{Sex bias in
%%       graduate admissions: Data from Berkeley,} \textit{Science}, 187, 398--403.
%% 
%%       \bibitem[{Cochran(1972)}]{cochran:1972}
%%       Cochran, W.~G. (1972), \enquote{Observational Studies,} in \textit{Statistical
%%       Papers in Honor of George Snedecor}, Iowa State University Press, pp. 77--90.
%% 
%%       \bibitem[{Hansen(2004)}]{hansen:2004}
%%       Hansen, B.~B. (2004), \enquote{Full matching in an observational study of
%%         coaching for the {SAT},} \textit{Journal of the American Statistical
%%       Association}, 99, 609--618.
%% 
%%       \bibitem[{Hansen(2008)}]{hansen:statmed:2008}
%%       --- (2008), \enquote{The essential role of balance tests in propensity-matched
%%         observational studies: Comments on ``A critical appraisal of propensity-score
%%         matching in the medical literature between 1996 and 2003'' by Peter Austin,
%%       Statistics in Medicine.} \textit{Statistics in Medicine}, 27, 2050--2054.
%% 
%%       \bibitem[{Hansen and Bowers(2008)}]{hansen:bowers:2008}
%%       Hansen, B.~B. and Bowers, J. (2008), \enquote{Covariate balance in simple,
%%       stratified and clustered comparative studies,} \textit{Statistical Science},
%%       23, 219--236.
%% 
%%       \bibitem[{Hansen and Klopfer(2006)}]{hansen:klopfer:2006}
%%       Hansen, B.~B. and Klopfer, S.~O. (2006), \enquote{Optimal full matching and
%%         related designs via network flows,} \textit{Journal of Computational and
%%       Graphical Statistics}, 15, 609--627.
%% 
%%       \bibitem[{Hill et~al.(2000)Hill, Rubin, and Thomas}]{hill:etal:2000}
%%       Hill, J., Rubin, D.~B., and Thomas, N. (2000), \enquote{The Design of the New
%%         York School Choice Scholarship Program Evaluation,} in \textit{Research
%%       Design: Donald Campbell's Legacy}, ed. Bickman, L., Sage Publications.
%% 
%%       \bibitem[{Imai et~al.(2008)Imai, King, and Stuart}]{imaiKingStuart:2007}
%%       Imai, K., King, G., and Stuart, E. (2008), \enquote{Misunderstandings among
%%         Experimentalists and Observationalists: Balance Test Fallacies in Causal
%%       Inference,} \textit{Journal of the Royal Statistical Society}, 171, 481--502.
%% 
%%       \bibitem[{Rosenbaum(2002)}]{rosenbaum:2002}
%%       Rosenbaum, P.~R. (2002), \textit{Observational Studies}, Springer-Verlag, 2nd
%%       ed.
%% 
%%       \bibitem[{Rubin(1997)}]{rubin:1997}
%%       Rubin, D.~B. (1997), \enquote{Estimating Causal Effects from Large Data Sets
%%       Using Propensity Scores,} \textit{Annals of Internal Medicine}, 127,
%%       757--763.
%% 
%%       \bibitem[{Stuart(2010)}]{stuart2010matching}
%%       Stuart, E.~A. (2010), \enquote{{Matching methods for causal inference: A review
%%       and a look forward},} \textit{Statistical Science}, 25, 1.
%% 
%%   \end{thebibliography}
%% 
%% }

\end{frame}

\end{document}
