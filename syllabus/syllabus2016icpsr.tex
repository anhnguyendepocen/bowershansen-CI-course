\documentclass[12pt]{article}

%\usepackage{epsf,graphicx}
\usepackage[T1]{fontenc}
\usepackage{mathptmx}
\usepackage[top=1in, bottom=1in, right=1in, left=1in]{geometry}
\usepackage{natbib}
\usepackage{bibentry} %% to do bib in syllabus
\newcommand{\bibverse}[1]{\begin{verse} \bibentry{#1}. \end{verse}}
\usepackage{hyperref}
\usepackage{parskip}

\usepackage{setspace}
\setcounter{secnumdepth}{0}

\title{Causal Inference for the Social Sciences}

\date{2016 ICPSR Summer Program, Session 2\\ 
(Version posted \today)
} %misnomer, as of now file only carries date
\author{ Jake Bowers\thanks{ Political Science and Statistics
    Departments, U. of Illinois, Champaign-Urbana; jwbowers@illinois.edu}
\and Ben Hansen\thanks{Statistics Department and Survey Research
  Center, U. of Michigan; ben.hansen@umich.edu} \and Tom
Leavitt\thanks{Political Science, Columbia U.; \mbox{thomasleavitt87@gmail.com}}\,  (t.a.)}


\begin{document}

\maketitle
\begin{abstract}
This course introduces methods and concepts used to infer causal
effects from comparisons of intervention and control groups.  Random
assignment plays a central role in the relevant statistical theories,
but the techniques and principles that emerge apply to nonrandomized
comparisons as well. We'll use the potential outcomes framework of
causality to analyze both randomized and observational studies,
distinguishing different forms of random assignment and separating
observational studies that involve instruments, discontinuities and
other devices, highlighting relative strengths of the designs and the
implications of study design for statistical analysis.  Propensity
score matching is treated in depth, with explicit instruction in the
use of ``optmatch'' and related packages in R; other areas of
methodological focus include assessment of covariate balance by
specification tests and other methods; inference methods that are a
robust to small sample sizes, weak instruments, spillover and interaction effects, heterogeneous treatment effects,
and/or misspecification of response surfaces; and omitted variable sensitivity
analysis.  %The content is geared specifically toward students and researchers in the social sciences, with examples are drawn from economics, political science, public health, and sociology, among other fields.

The course presupposes knowledge of multiple regression at the level
of the ICPSR course Regression: II, as well as multiple regression
with binary dependent variables (as taught in the ICPSR courses
Regression: III or Maximum Likelihood).  The part of the course presenting
matching requires the use of R for computation, but other methods presented in
the course are readily implemented either in R or in Stata.
\end{abstract}

% \section{Meetings}

The course meets 3--5pm, Monday through Friday, from July 24 through
August 18, in Weiser Hall, Room 296. (On the second floor of the
low-rise part of Weiser Hall, at the southernmost point in the
building.)  Ben is primary instructor for the first two weeks, Jake
for the second.

Tom holds office hours 12--3 pm, in 1480 Thompson. Ben and/or Jake
have office hours Monday, Wednesday and Friday between 9:30 and 10:30
a.m., in the Perry Building at room 2342 (Ben's office) or room 2333
(Jake's office).  Additional office hours may be announced.


\clearpage

\nobibliography*


\section{Overview}
We may all warn our freshmen that association is not causation, but inferring causation has always been a central aim both for statisticians and for their collaborators. Until recently, however, inference of causation from statistical evidence depended on murky, scarcely attainable requirements; in practice, the weight of casual arguments was largely determined by the scientific authority of the people making them.

Requirements for causal inference become more clear when they are
framed in terms of \emph{potential outcomes}.  This was first done by
Neyman, who in the 1920s used potential outcomes to model agricultural
experiments.  Fisher independently proposed a related but distinct,
ultimately more influential, analysis of experiments in 1935, and a
rich strain of causal analysis developed among his intellectual
progeny.  It clarified the differing requirements for causal inference
with experiments and with observational data, isolating the distinct
contributions required of the statistician and of his disciplinary
collaborators; generated more satisfying methods with which to address
potential confounding due to measured variables; qualitatively and
quantitatively advanced our grasp of unmeasured confounding and its
potential ramifications; furnished statistical methods with which to
eke more out of the strongest study designs, under fewer assumptions;
and articulated principles with which to understand study designs as a
spectrum, rather than a dichotomy between ``good" experiments and
``bad" observational studies. Understanding the methods and outlook of
the school founded by Fisher's student W.G. Cochran will be the
central task of this course.

The course begins by applying the Fisher and Neyman-Rubin models of
causality to randomized experiments, touching on considerations
specific to clustered treatment assignment, ``small'' sample sizes and
treatment effect heterogeneity.  The next segment addresses conceptual
and methodogical challenges of applying the same models to analysis of
non-experimental data. This course segment covers ignorability,
selection, ``common support,'' covariate balance, paired comparisons,
optimal matching and propensity scores. A short separate section
introduces another method aiming to identify experiment-like
structures in observational data, namely regression discontinuity,
before a return to experiments.

With these foundations in place, the course's second half adds
conceptual depth and methodological flexibility. 
% JAKE DO YOU AGREE?  1ST STAB HERE AT CODIFYING WHAT WE WERE TALKING
% ABOUT THE OTHER DAY...
Central topics include instrumental variables and local average
treatment effects, stratified designs with clustering, interference,
omitted variable sensitivity analysis and adapting workhorse
techniques such as multiple regression to the demands of causal
inference.  Over the course of the four weeks the course becomes
progressively less lecture-oriented and more hands-on, with increasing
emphasis on computing strategies in R.

\section{Administrative}

\subsection{Textbook(s)}

The main text for the course is 

\bibverse{rosenbaum10book}
(Hereafter ``\textit{DOS}''.)  Although we won't follow it closely,
its goals and methods align with the course's, and it will be useful
as a reference and supplement. Other texts that we draw on include

\bibverse{gerber2012field}

\bibverse{dunning2012natural}

\bibverse{imbensRubin15}

\bibverse{keele15CIreview}
% If you'll be using experiments, propensity scores, regression
% discontinuity, or instrumental variables in your work, then each of
% these is a worthwhile investment.

A number of other graduate-level texts on or touching on causal inference have emerged in recent
years.  Many of them are quite good, including:\\

\bibverse{angrist2009mostly}

\bibverse{berk2004regression}

\bibverse{gelman2006dau}

\bibverse{morgan2007counterfactuals}

\bibverse{murnane2010methods}


Other readings will be assigned and distributed electronically through
a password-protected ``Ctools'' Web site
(\hyperref{https://ctools.umich.edu/portal/site/ef6d7adf-4545-4a10-bada-42b5b6bb9ed3}{``ICPSR Causal Inference'16''}).  
 (To get access to this you'll have to set up a University of Michigan ``Friend'' account, then share the email address associated with this friend account with one of the course instructors.)


If you're new to R, we suggest getting hold of:

\bibverse{fox2011r}

R software will be required for several specific segments of the course.  With
some independent effort, students not familiar with R in advance should be
able to learn enough R during the course to complete these assignments. We
also recommend some work with R --- for example, via working through some
online R courses --- before the course for students who have never
used it before.

 \subsection{Free accounts you should sign up for}

 \begin{itemize}
 \item University of Michigan Friend account:\\
\url{www.itcs.umich.edu/itcsdocs/s4316/}\\
Unless you already have an email address ending in
\texttt{umich.edu}, and you'd like to receive course messages at that
address --- in this case, let us know your name and what that address
is.  Otherwise, sign up for a Friend account at your soonest convenience, using the email address
you gave to the summer program. 
\item Github, \url{github.com/join} . (Or, if you prefer, Open Science
  Framework, \url{osf.io} .)
 \end{itemize}
% Readings will be distributed via the 
% %``ICPSR Causal Inference'13'' course site within the
% \href{http://ctools.umich.edu}{ctools.umich.edu} system.  
% % Online questions
% % and discussions about course content will be managed through
% % \href{piazza.com/umich/summer2014/icspr/home}{piazza.com}.  
% % We won't use these
% % systems during lectures, although from time to time there'll be
% % electronic polls that you can participate in at
% % \href{http://PollEv.com/bbh}{PollEv.com/bbh} or by text message.
% % Add site names!

% You'll
% receive invitations to join both sites at the email address that
% appears for you on the course roster, the address you gave when you
% signed up for the Summer Program.  Sign in to the sites
% using that email address.  (You may have received a \texttt{umich}
% email address upon arrival in Ann Arbor, perhaps for using the
% University's wireless networks.  Do \textit{not} use this address for
% accessing Ctools: if you do, you may be allowed to
% authenticate to the system but you won't be able to see or access website for this course!)
% %\subsection{Important note.}  Students likely to solicit recommendation letters to graduate programs from their ICPSR instructors are advised to take other ICPSR courses.  

\subsection{Assignments}
%Computer or pencil and paper exercises will be assigned periodically and collected on Mondays and on Thursdays at the beginning of class. 
Assignments are due each Tuesday, at the beginning of
class. Parts of the assignment will be given at the beginning of the
week, but other parts will be given during class, over the course of
the week.  Many of these daily assignments will be given with the
expectation that they'll be completed by the next course meeting,
although they'll only be collected at the end of the week.
Late homework will not be accepted without cause (or prior arrangement
with the teaching assistant).   

You're welcome to submit a paper at the end of the course, whether or not you're taking the course for credit.  In that case we'll return it with comments within a month or so of the course's completion.  (If you're taking the course for a grade, the paper won't contribute to the grade unless you're on  the borderline between two grades.) 

Participation is expected.  It can take various forms:
\begin{enumerate}
\item Do in-class exercises and discuss them with your peers;
\item Occasionally make a comments, raise a question or offer a clarification
  during lecture;
\item Contribute to in-class discussions.
\item \label{it:part0} Use a github pull request to suggest a clarification or other enhancement to
  a course slide or worksheet.
\item \label{it:part1} Drop by one of the professor's office hours to share a point
  that you \textit{and at least one classmate} would like to have clarified or
  amplified, or to point out a connection to your field; 
\item \label{it:part2} Give a 5-10 minute in-class presentation of a paper in your
  field that uses methods or designs we're discussing in the course.
\end{enumerate}
If you are taking the course for a grade, make a point of doing at
least one of \ref{it:part0}, \ref{it:part1} and \ref{it:part2}.  There'll be an
electronic sign-up for \ref{it:part2}.

% Course participants will be expected either to give a presentation or to submit a short paper about assigned readings related to their fields of study.  Presentations will be given alone or in pairs, and last 10-20 minutes; short papers should be 5-8 pages long.  Students must decide no later than the end of the second week whether they’ll present or submit a paper, and on what topic; presentations will be given at times designated by the instructor, whereas papers must be submitted by the beginning of the fourth week.


\section{Course contents}

\subsection{Random assignment as a basis for inference}

\subsubsection{Experiments and the foundations of causal inference}

Medical- and social-science data generating processes can be difficult to
capture accurately in a single regression equation, for various reasons.  
The statistical foundations of randomized experiments are much more
satisfying, particularly when they are taken on their own
terms. Fisher and Neyman did this earlier in the 1920s and 30s, in
work that Rubin, Holland and others reinvigorated beginning in the
1970s.  The course begins by surveying the circle of ideas to emerge
from this. 

% In partial compensation for this early focus on experiments to the
% expense of observational studies, for the course's
% first meeting please read  \textit{DOS,}  pages 3--4.  It's short and it
% contains a rationale for beginning by understanding experiments, even
% if you'll be working exclusively with observational data. 

Some readings for this course segment:
%\begin{itemize}
\begin{verse} Section~1.2, ``Experimentation defined,''  of 
\bibentry{kinder1993experimental}.  (Particularly pp. 5--10.) \end{verse}

\begin{verse}\bibentry{holland:1986a}, Sections~1--4. (The article that brought the ``Rubin 
  Causal Model'' to  statisticians' attention.)\end{verse}

\begin{verse}
  Chapter 1 of DOS.
\end{verse}


\subsubsection{Inference for causal effects: the Neyman tradition}

\begin{verse}
  Chapters~1, 2 of \bibentry{gerber2012field}.
\end{verse}

\begin{verse} Endnote spanning pages A-32 and 33,
  \bibentry{freedman:purv:pisa:1998}.  (This can be read as a pr{\'e}cis
  of: \bibentry{neyman:1990}.)\end{verse}


\subsubsection{Inference for causal effects: the Fisherian tradition}
\begin{verse}
  Chapter 2 of \bibentry{rosenbaum2010design}.
\end{verse}

\bibverse{rosenbaum:1996:onAIR} (A Fisherian approach to instrumental variables.) 



Additional reading:

\begin{verse}
  Section 5.3, ``Instruments,'' of \bibentry{rosenbaum2010design}.
\end{verse}

\begin{verse}
  Section 2.3 of \bibentry{rosenbaum:2002a}.
\end{verse}

\begin{verse} Chapter~1, ``Introduction,'' of \bibentry {fisher:1935}.\end{verse}



\subsection{Found experiments}


\subsubsection{Principled search for designs with ignorable assignment}

\begin{verse}
  Chapter 1 of DOS.
\end{verse}

\begin{verse}
  Chapter 6 of \bibentry{dunning2012natural}.
\end{verse}

\bibverse{hansenSales2015cochran}
\bibverse{hansen:bowers:2008}



\subsubsection{Propensity score methods}

Motivation and background:

\bibverse{rubin:1979}

\bibverse{rubinWaterman06}% PS matches as "clones" notion 

\bibverse{ho:etal:07}


\bibverse{rosenbaum:2001b}% Correcting the cloning mistake

\bibverse{imbens:rubin:2008palgrave}

A success and a failure:

\bibverse{bifulco2012can}

\bibverse{arceneaux2010cautionary}



\subsubsection{Matching a focal group to controls}
%\begin{itemize} 
\begin{verse}\textit{DOS}, Chap 8--9, 13.\end{verse} 
\begin{verse}\bibentry{rosenbaum:rubi:1985a}.\end{verse}
\begin{verse}\bibentry{ho:etal:07}.\end{verse}
\begin{verse}\bibentry{hansen2009b}.\end{verse}
%\end{itemize}


\subsubsection{Regression Discontinuity Designs}

\bibverse{lee08}
\bibverse{mccrary2008manipulation}
\bibverse{causek11}
\bibverse{cattaneo2014randomization}
\bibverse{salesHansen2015RIforRD}

\subsection{Causal inference from assignment mechanism models}


\subsubsection{Instrumental Variables}
\bibverse{AIR:96}
\bibverse{lityau98}
\bibverse{baiocchi2010building} 
\bibverse{nickerson2008voting}
\bibverse{gerber2010baseline}


\subsubsection{Modeling the assignment mechanism}

Why we might need a special story or additional steps in order to use familiar models:
\bibverse{freedman2008} 
but see also
\bibverse{lin2013agnostic}

A special story and a few additional steps that enable you to use familiar models:
\bibverse{rosenbaum:2002a} 


\textbf{In addition:}

Example involving hierarchical data and a rather elaborate logistic regression model.  See in particular methodological appendix.
\bibverse{cerdaetal2011} 

Large samples enable some useful trickery.  Example also demonstrates use of clustering.
\bibverse{bowers:hans:2008}

Instrumental variables.
\bibverse{rosenbaum:1996:onAIR}

\bibverse{imbens:rose:2005}



\subsection{Last but not least}
\subsubsection{Sensitivity analysis}

\begin{verse} \textit{DOS}, Chap 3 \end{verse}

\bibverse{hosmanetal2010}

\subsubsection{Interference}

\bibverse{Sobel:what:2006}

\bibverse{hudgens2008toward}

\bibverse{rosenbaum2007ibu}

\bibverse{bowersFP2013}

\bibverse{ichino2012deterring}

%\section{Additional Notes}
\clearpage  
\bibliographystyle{plain}
\nobibliography{ci}
%\nobibliography{master,abbrev_long,causalinference,biomedicalapplications,misc}
%\nobibliography{../2013/BIB/master,../2013/BIB/abbrev_long,../2013/BIB/causalinference,../2013/BIB/biomedicalapplications,../2013/BIB/misc}
%"master" is Ryan Moore's file (as of summer 2012).
%"causalinference" is Ben's, and lives (as of summer 2013) in
% http://dept.stat.lsa.umich.edu/~bbh/texmf/bibtex/bib/


\end{document}
