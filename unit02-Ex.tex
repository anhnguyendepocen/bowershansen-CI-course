\documentclass{article}
\usepackage{natbib}
\usepackage{wrapfig}
\title{In-class exercises on instrumental variables (IV) concepts and methods }
\date{ICPSR Session 2 (\today)}

\usepackage{icpsr-classwork}

\begin{document}
\maketitle

\begin{minipage}{1.0\linewidth}

\begin{wrapfigure}{r}{.65\linewidth}
\igrphx{albertsonLawrence2009tab1}
\end{wrapfigure}

The table at right comes from Albertson and Lawrence's (2009, \textit{Amer. Pol. Res.} \textbf{37}/2) paper ``After the credits roll: The long term effects of educational television on public knowledge and attitudes,'' which reported on two field experiments.   Besides being thematically related, the two experiments followed very similar designs, using random digit dialing (RDD) telephone sampling to recruit initial samples, then employing a randomized encouragement design.  One can analyze them using IV methods under the following assumptions:

\begin{itemize}
\item Random assignment: $\mathrm{Pr}(Z_{i}=1) = $ constant over study population;  and $\mathrm{Z} $ is determined by coin tosses, card shuffles or equivalent methods
\item Monotonicity (no defiers)
\item Excludability (for all $z$, $d$, $y_{Z=z, D=z} \equiv y_{D=d}$)
\item $p_{c} = \mathrm{Pr}(\mathrm{Complier}) > 0$ 
\end{itemize}
  
\end{minipage}
\vspace{4ex}

\begin{enumerate}
\item \label{item:1}Assuming (for this question) that the IV assumptions are valid as applied to the Round 2 respondents of each experiment, estimate the ACE of encouragement to view the program on program viewing. (If you're working with a partner, divide the two experiments between you, and explain your answers to one another when you're done.)
\item Describe a hypothetical outcome variable that you would have collected, for the/one of the experiments you answered~(\ref{item:1}) about,  had you written the study questionnaire. Posit a hypothetical value for your estimated ACE of encouragement on this outcome; combine this and your answer to (\ref{item:1}) to furnish a (hypothetical) Bloom-type estimate of the complier average causal effect (CACE).  
\item Response rates for the Round 1 RDD sample recruitment effort are not reported in the table. Would you need to know these response rates in order to evaluate the IV assumptions as applied to a ``study population'' of Round 1 respondents?  As applied to a study population of Round 2 respondents? 
\item Would you need to know the Round 1 response rates in order to evaluate the generalizability of each experiments's findings to the populations described in ``Location of broadcast''?  Would they be relevant?  
\item Whether a subject responds at round 2, $R$, can be considered as a (secondary) outcome. Let $r_{Ti}$ be an indicator of whether subject $i$ would respond at round 2 if assigned to the encouragement condition, with $r_{Ci}$ denoting $i$'s corresponding potential response if assigned to control.
  \begin{enumerate}
  \item Explain why the Random Assignment assumption would be violated for a study population consisting of \textit{Round 2} respondents, despite proper use of random assignment procedures during Round 1, if it's common that $r_{T}=1$ while $r_{C}=0$.
  \item Conversely, explain how for a study population of Round 2 respondents, Random Assignment follows from a combination of Random Assignment at round 1 plus an additional assumption that $r_{Ti} \equiv r_{Ci}$, all $i$. 
  \item What pieces of information that are necessary to appraise the plausibility of Albertson and Lawrence's (2009) random assignment assumption are missing from the table? 
  \end{enumerate}

\end{enumerate}
\end{document}
