%  For slides only
%%% Usage:
%% $ cat slidesonly.tex unitXX-foo.tex | pdflatex
%% $ pdfjam -o unitXX-slides{a/b/...}.pdf texput.pdf '1,{1stpage}-{lastpage}'

\documentclass{beamer} 
\newcommand{\tcw}{\textcolor{black}}
\newcommand{\mynoteonly}{}
\newcommand{\nottheirhandout}{handout:0}

% % For handout
%% Usage:
%% $ cat handout.tex unitXX-foo.tex | pdflatex
%% $ pdfjam -o unitXX-handout{a/b/...}.pdf texput.pdf '{1stpage}-{lastpage}'
\documentclass[handout]{beamer}
\usepackage{pgfpages}
\pgfpagesuselayout{4 on 1}[letterpaper,border shrink=5mm]%\nofiles

\mode<handout>{\setbeamercolor{background canvas}{bg=black!5}}
\newcommand{\tcw}{\textcolor{structure.bg}}
\newcommand{\mynoteonly}{| handout:0}
\newcommand{\nottheirhandout}{handout:0}

%For handout + mynotes
%%% Usage:
%% $ cat handout+mynotes.tex unitXX-foo.tex | pdflatex
%% $ pdfjam --nup 1x4 -o unitXX-foo-wmn{a/b/...}.pdf texput.pdf '1,{1stpage}-{lastpage}'
\documentclass[handout]{beamer}
\usepackage{pgfpages}
%%\pgfpagesuselayout{2 on 1}[letterpaper,border shrink=5mm] 
\setbeameroption{show notes on second screen=left}
\newcommand{\tcw}{\textcolor{black}}
\newcommand{\mynoteonly}{}
\newcommand{\nottheirhandout}{}

\newcommand{\igrphx}[2][width=\linewidth]{\includegraphics[#1]{images/#2}}

\renewcommand{\strut}{\rule{0pt}{3ex}}

% Frame note commands, with time budget updating
\newcounter{timeTotal}
\newcommand{\printUpdateTimeTotal}[1]{ \addtocounter{timeTotal}{#1}
  \textrm{(#1 min budgeted)} \hfill \textbf{Finish by
    \arabic{timeTotal}\ min from start}\\[1ex]}
\newcommand{\tnote}[3][10]{#2 \note{#2: #3 \\[.5ex]} \printUpdateTimeTotal{#1}}
\newcommand{\itnote}[2][10]{\note{ \begin{itemize} #2 \end{itemize}
\vfill \noindent  \printUpdateTimeTotal{#1} }}
\newcommand{\ennote}[2][10]{\note{ \begin{enumerate} #2 \end{enumerate}
\vfill \noindent  \printUpdateTimeTotal{#1} }}
\newcommand{\Note}[2][10]{\note{ #2 \mbox{ }\\ \vfill \noindent  \printUpdateTimeTotal{#1} }}

\newenvironment{Column}[1][.5\linewidth]{\begin{column}{#1}}{\end{column}}

\mode<handout>{\beamertemplatesolidbackgroundcolor{black!5}} 
\mode<article>{\usepackage{fullpage}}
\mode<presentation>
{
  \usetheme{Boadilla}
  % or ...

  \setbeamercovered{transparent}
  % or whatever (possibly just delete it)
}
\usepackage{url}
\usepackage{ulem}

\usepackage[english]{babel}
% or whatever

\usepackage[latin1]{inputenc}
% or whatever

\usepackage{times}
\usepackage[T1]{fontenc}
% Or whatever. Note that the encoding and the font should match. If T1
% does not look nice, try deleting the line with the fontenc.

\AtBeginSubsection[]
{
  \begin{frame}<beamer>
    \frametitle{Outline}
    \tableofcontents[subsectionstyle=show/shaded/hide]
  \end{frame}
}

\AtBeginSection[]
{
  \begin{frame}<beamer>
    \frametitle{Outline}
    \tableofcontents[hideothersubsections,sectionstyle=show/shaded]
  \end{frame}
}


% If you wish to uncover everything in a step-wise fashion, uncomment
% the following command: 

%\beamerdefaultoverlayspecification{<+->}

%USEFUL CODE TEMPLATES:
%\begin{itemize}[<+-| alert@+>]


% \begin{frame}[fragile]


% \begin{columns}
% \column{.4\textwidth}  
%   \begin{itemize}
%   \item<1-| alert@1> Why sample twice?
%   \item<2-| alert@2> Illustrative example.
%   \item<3-| alert@3> PSU's and SSU's.
%   \item<4-| alert@4> Stratification in combination with two-stage cluster sampling.
%   \item<5-| alert@5> Detailed example.
%   \end{itemize}
% \column{.6\textwidth}
% \only<2| handout:0>{
% \includegraphics[width=\textwidth]{nursingHomes}}
% \only<5| handout:1>{
% \includegraphics[width=\textwidth]{hosp_cover}}
% \end{columns}

% FOR INCLUDING R CODE. 
%\usepackage{listings}
%\lstset{language=R}
% \begin{frame}[fragile]
% \frametitle{Some code}
% \begin{lstlisting}
% > plot(myobj)
% > rm(myobj)
% \end{lstlisting}  
% \end{frame}

%\addtocounter{framenumber}{-1}


\usepackage{amsmath,amsthm}
\usepackage{wasysym,pifont}
\usepackage{Sweave}
\usepackage{ulem}
\usepackage{textcomp}
\usepackage{versions}
%% amsthm-type theorem environment specifications -- 
%% see amsthdoc.pdf in amscls documentation
\theoremstyle{plain}
\newtheorem{prop}{Proposition}[section]
\newtheorem{lem}[prop]{Lemma}

\newtheorem*{thm}{Proposition}
\newtheorem*{cor}{Corollary}

\theoremstyle{definition}
\newtheorem{defn}{Definition}[section]

\newcommand{\Pdistsym}{P}
\newcommand{\Pdistsymn}{P_n}
\newcommand{\Qdistsym}{Q}
\newcommand{\Qdistsymn}{Q_n}
\newcommand{\Qdistsymni}{Q_{n_i}}
\newcommand{\Qdistsymt}{Q[t]}
\newcommand{\dQdP}{\ensuremath{\frac{dQ}{dP}}}
\newcommand{\dQdPn}{\ensuremath{\frac{dQ_{n}}{dP_{n}}}}
\newcommand{\EE}{\ensuremath{\mathbf{E}}}
\newcommand{\EEp}{\ensuremath{\mathbf{E}_{P}}}
\newcommand{\EEpn}{\ensuremath{\mathbf{E}_{P_{n}}}}
\newcommand{\EEq}{\ensuremath{\mathbf{E}_{Q}}}
\newcommand{\EEqn}{\ensuremath{\mathbf{E}_{Q_{n}}}}
\newcommand{\EEqni}{\ensuremath{\mathbf{E}_{Q_{n[i]}}}}
\newcommand{\EEqt}{\ensuremath{\mathbf{E}_{Q[t]}}}
\newcommand{\PP}{\ensuremath{\mathbf{Pr}}}
\newcommand{\PPp}{\ensuremath{\mathbf{Pr}_{P}}}
\newcommand{\PPpn}{\ensuremath{\mathbf{Pr}_{P_{n}}}}
\newcommand{\PPq}{\ensuremath{\mathbf{Pr}_{Q}}}
\newcommand{\PPqn}{\ensuremath{\mathbf{Pr}_{Q_{n}}}}
\newcommand{\PPqt}{\ensuremath{\mathbf{Pr}_{Q[t]}}}
\newcommand{\var}{\ensuremath{\mathbf{V}}}
\newcommand{\varp}{\ensuremath{\mathbf{V}_{P}}}
\newcommand{\varpn}{\ensuremath{\mathbf{V}_{P_{n}}}}
\newcommand{\varq}{\ensuremath{\mathbf{V}_{Q}}}
\newcommand{\cov}{\ensuremath{\mathbf{Cov}}}
\newcommand{\covp}{\ensuremath{\mathbf{Cov}_{P}}}
\newcommand{\covpn}{\ensuremath{\mathbf{Cov}_{P_{n}}}}
\newcommand{\covq}{\ensuremath{\mathbf{Cov}_{Q}}}

\newcommand{\hatvar}{\ensuremath{\widehat{\mathrm{Var}}}}
\newcommand{\hatcov}{\ensuremath{\widehat{\mathrm{Cov}}}}

\newcommand{\sehat}{\ensuremath{\widehat{\mathrm{se}}}}

\newcommand{\combdiff}[1]{\ensuremath{\Delta_{{z}}[#1]}}
\newcommand{\Combdiff}[1]{\ensuremath{\Delta_{{Z}}[#1]}}

\newcommand{\psvec}{\ensuremath{\varphi}}
\newcommand{\psvecgc}{\ensuremath{\tilde{\varphi}}}


\newcommand{\atob}[2]{\ensuremath{#1\!\! :\!\! #2}}
\newcommand{\stratA}{\ensuremath{\mathbf{S}}}
\newcommand{\stratAnumstrat}{\ensuremath{S}}
\newcommand{\sAsi}{\ensuremath{s}}

\newcommand{\permsd}{\ensuremath{\sigma_{\Pdistsym}}}
% \newcommand{\dz}[1]{\ensuremath{d_{z}[{#1}]}}
\newcommand{\dZ}[1]{\ensuremath{d_{Z}[{#1}]}}
\newcommand{\tz}[1]{\ensuremath{t_{{z}}[#1]}}
\newcommand{\tZ}[1]{\ensuremath{t_{{Z}}[#1]}}


\newlength{\tabcolsepadj}
\setlength{\tabcolsepadj}{1.3mm}

%%% NEWBLOCK UNDEFINED BUG
\def\newblock{\hskip .11em plus .33em minus .07em}

%%% tightlist undefined control sequence bug 
%%% http://tex.stackexchange.com/questions/257418/error-tightlist-converting-md-file-into-pdf-using-pandoc
\providecommand{\tightlist}{%
  \setlength{\itemsep}{0pt}\setlength{\parskip}{0pt}}
\newcommand{\igrphx}[2][width=\linewidth]{\includegraphics[#1]{images/#2}}

\renewcommand{\strut}{\rule{0pt}{3ex}}

\newenvironment{Column}[1][.5\linewidth]{\begin{column}{#1}}{\end{column}}

\usetikzlibrary{arrows} % also see \tikzstyle spec below after \begin{doc}

\newcommand{\mlpnode}[1]{\tikz[baseline=-.5ex] \coordinate (#1) {};}
%\newcommand{\mlpnode}[1]{\raisebox{.5ex}{\pnode{#1}}}


\usepackage{xspace}

\includeversion{pedantic} % \excludeversion

\tikzstyle{every picture}+=[remember picture]



\title{Unit 4: Baseline comparability and blocking}
% \author, date moved to beamer-preamble-*-all.tex

\begin{document}


  \begin{frame}
    \frametitle{Outline \& Readings}

\tableofcontents[subsectionstyle=show/hide/hide]

  % \alert{Readings for this Unit:}\\
  % \begin{itemize}
  % \item Rosenbaum (2002), ``Covariance adjustment in randomized experiments and observational  studies, '' \S 2.3 ({\em Statistical Science}, 17/3:286--327).
  % \end{itemize}

\input{announcement-of-the-day}  % announcement-of-the-day.tex not
                                % part of repo

\end{frame}
\itnote{
%Interesting template possibility: examples, assumptions, issues
\item examples: UCBgrads/UMfac vs KC03; Medellin
}
\section[tests of assignment]{Testable manifestations of biased and random assignment}

\begin{frame}[fragile]
  \frametitle{1971 UC Berkeley Admissions}

Bickel \textit{et al.}'s (1975)%\citeyearpar{Bickel:etal:1975} 
famous data set:\\[1ex]

% latex table generated in R 3.2.2 by xtable 1.8-0 package
% Fri Dec  4 01:28:36 2015
\begin{tabular}{rrr}
  \hline
 & Admitted & Rejected \\ 
  \hline
Male & 0.45 & 0.55 \\ 
  Female & 0.30 & 0.70 \\ 
   \hline
\end{tabular}
\vspace{2ex}

Discrimination?\\[3ex]

\pause
\begin{Schunk}
\begin{Soutput}
	Pearson's Chi-squared test with Yates'
	continuity correction

data:  ucb1971
X-squared = 90, df = 1, p-value <2e-16
\end{Soutput}
\end{Schunk}
\end{frame}

\begin{frame}[fragile]
  \frametitle{1971 UC Berkeley Admissions}


Bickel \textit{et al.}'s (1975)%\citeyearpar{Bickel:etal:1975} 
famous data set:\\[1ex]

% latex table generated in R 3.2.2 by xtable 1.8-0 package
% Fri Dec  4 01:28:36 2015
\begin{tabular}{rrr}
  \hline
 & Admitted & Rejected \\ 
  \hline
Male & 0.45 & 0.55 \\ 
  Female & 0.30 & 0.70 \\ 
   \hline
\end{tabular}
\vspace{2ex}

Discrimination? ---Perhaps not.\\[3ex]

Proportions of men and women applicants accepted, \textit{by department}, for 6 large departments.

% latex table generated in R 3.2.2 by xtable 1.8-0 package
% Fri Dec  4 01:28:36 2015
\begin{tabular}{rrrrrrr}
  \hline
 & A & B & C & D & E & F \\ 
  \hline
Male & 0.62 & 0.63 & 0.37 & 0.33 & 0.28 & 0.06 \\ 
  Female & 0.82 & 0.68 & 0.34 & 0.35 & 0.24 & 0.07 \\ 
   \hline
\end{tabular}
\end{frame}
\itnote{
 \item What gives?  Look for a while and see you can see what's going on.  Not rocket science, but tricky.  Have to guess a little about stuff that's not evident from this table.
\item {}\textit{[time permitting]} Now try to describe it to your neighbor.   
\item {}\textit{[time permitting]} Anyone care to give a succinct explanation?
}

\begin{frame}<1>[fragile,label=ucbBalFr] 
  \frametitle{1971 UC Berkeley Admissions}

\bigskip

Proportions of men and women \textit{applying} to the 6 departments.\\
% latex table generated in R 3.2.2 by xtable 1.8-0 package
% Fri Dec  4 01:28:36 2015
\begin{tabular}{rrrrrrrr}
  \hline
 & A & B & C & D & E & F & Overall \\ 
  \hline
Male & 31 & 21 & 12 & 15 & 7 & 14 & 100 \\ 
  Female & 6 & 1 & 32 & 20 & 21 & 19 & 100 \\ 
  diff. & 25 & 19 & -20 & -5 & -14 & -5 & 0 \\ 
   \hline
\end{tabular}\\
\vfill
\end{frame}

\note[itemize]{
\item Recall that A--F are in descending order of ease of admission.
}


\begin{frame}[fragile] 
  \frametitle{1971 UC Berkeley Admissions}

\bigskip

Proportions of men and women \textit{applying} to the 6 departments.\\
% latex table generated in R 3.2.2 by xtable 1.8-0 package
% Fri Dec  4 01:28:36 2015
\begin{tabular}{rrrrrrrr}
  \hline
 & A & B & C & D & E & F & Overall \\ 
  \hline
Male & 31 & 21 & 12 & 15 & 7 & 14 & 100 \\ 
  Female & 6 & 1 & 32 & 20 & 21 & 19 & 100 \\ 
  diff. & 25 & 19 & -20 & -5 & -14 & -5 & 0 \\ 
   \hline
\end{tabular}\\
\bigskip

Define dummy variables \texttt{DeptA}, \ldots, \texttt{DeptF}. Comparison is \textit{balanced} for \texttt{DeptA} if M, W means of \texttt{DeptA} are similar.


\vfill
\mbox{ }

\end{frame}

\note[itemize]{
%\item discuss conjectures from previous slide
%\item Male/Female differences on these percentages as ANOVA contrasts, or as OLS regression coefficients.
\item The results suggest it isn't OK to analyze ignoring department
  --- at least, if department matters for likelihood of
  admission. Pretty clear that it should --- it's a ``potential
  confounders'';
\item what's not clear yet is whether the difference we're
  looking at might be a fluke of chance, similar to flukes we might
  have seen had we been able to assign gender at random.
\item Let's consider how this finding is reflected in terms of
  goodness-of-fit tests of models. 
}

\begin{frame}[fragile,label=UCBAttestFr] 
  \frametitle{1971 UC Berkeley Admissions}
  \framesubtitle{$t$-tests to check balance}

\bigskip

Proportions of men and women \textit{applying} to the 6 departments.\\
% latex table generated in R 3.2.2 by xtable 1.8-0 package
% Fri Dec  4 01:28:36 2015
\begin{tabular}{rrrrrrrr}
  \hline
 & A & B & C & D & E & F & Overall \\ 
  \hline
Male & 31 & 21 & 12 & 15 & 7 & 14 & 100 \\ 
  Female & 6 & 1 & 32 & 20 & 21 & 19 & 100 \\ 
  diff. & 25 & 19 & -20 & -5 & -14 & -5 & 0 \\ 
   \hline
\end{tabular}\\
\bigskip

% UCBAdefinition chunk included in pressetup.R

Define dummy variables \texttt{DeptA}, \ldots, \texttt{DeptF}. Comparison is \textit{balanced} for \texttt{DeptA} if M, W means of \texttt{DeptA} are similar. One way to check this:\\
\begin{Schunk}
\begin{Sinput}
> t.test(Dept=="A"~Gender, data=UCBA)
\end{Sinput}
\end{Schunk}
\begin{Schunk}
\begin{Soutput}
	Welch Two Sample t-test
data:  Dept == "A" by Gender 
t = 24, df = 4232, p-value < 2.2e-16    
\end{Soutput}
\end{Schunk}
\begin{Schunk}
\begin{Sinput}
> lm(Dept=="A"~Gender, data=UCBA)
\end{Sinput}
\end{Schunk}
\begin{Schunk}
\begin{Soutput}
             Estimate Std. Error t value Pr(>|t|)
(Intercept)      0.31     0.0074      41 1.6e-315
GenderFemale    -0.25     0.0117     -21  3.3e-95
\end{Soutput}
\end{Schunk}


\vfill
{\footnotesize \ldots where \texttt{UCBA} is (essentially)
  \texttt{as.data.frame(UCBAdmissions)}.}
\mbox{ }

\end{frame}
\itnote{
 \item Slight difference between $t$-stats is explained by the second
   assuming equal variances, which the first doesn't. }


\begin{frame}
  \frametitle{The Neyman-Rubin Model for (simple) experiments}
\vfill
  This is what randomization ensures:
$$ (\alert<4>{Y_t, Y_c},\alert<2>{X}) \perp \alert<3>{Z} $$
\vfill

\uncover<5->{I.e., each of $X$, $Y_{c}$ and $Y_{t}$is \alert<@+|
  handout:0>{balanced}} \uncover<6->{in expectation; modulo sampling
  variability.}
\vfill

\uncover<7->{You should test the part involving $X$ even before looking
  at outcomes (Cochran, 1965; Hansen \& Bowers, 2008; Hansen, 2008).}
  \uncover<8->{(For the opposite advice, see Senn, 1994; Imai et al 2008.)  }


\end{frame}
\note[itemize]{
  \item In controlled experiments, random assignment warrants this.
  \item In natural experiments, justified otherwise, or an article of
    faith (i.e. a model).
  \item In an experiment, the $x$es aren't necessary for inference.
  \item However, the part with the $x$es has testable consequences.}




\begin{frame}<\nottheirhandout>
\frametitle{Grow your own Simpson's paradox}
%\begin{columns}
% \begin{column}{.4\linewidth}
% \end{column}
\begin{center}%{.6\linewidth}
  UM-LSA Tenure-track faculty, by sex and
  division\footnote{Given in round numbers.  Actual totals: 629M,
    361W.  Source: ``Institutional indicators of Diversity: AY2010
    (Public),'' \textsc{Advance} committee, U. of Michigan.}
  \medskip
  
  \begin{tabular}{l|rrr|r} \hline
    & \multicolumn{1}{c}{Nat. Sci.} & \multicolumn{1}{c}{Soc. Sci.}
    &\multicolumn{1}{c}{ Humanities} & Overall\\ \hline
Men & 230 &250& 160 & 640\\[1.5ex]
Women & 60 &190& 110 & 360 \\[1.5ex] \hline
All & 290& 440 & 270 & 1000\\[2ex] 
\only<2->{Pub rate} & & & & \\ \hline
  \end{tabular}
\end{center}
%\end{columns}

\visible<2->{
Exercise: invent ``publication rates'' that are the same for men and
women in each division, but that lead to men having a higher
publication rate overall.
}    
\end{frame}


\againframe<\nottheirhandout>{UCBAttestFr}
\itnote{
\item Drawback \# 1: All hypothesis tests are bad, b/c we're not
  interested in whether the two populations differ. (This is what
  Senn, Imai et al are exercised about.)
 \item Drawback \# 2: t-test involves parametric assumptions, assumptions that don't work so great here. 
\item Drawback \# 3: doing a whole mess of t-tests is overly repetitive.
}

\begin{frame}[fragile] 
  \frametitle{1971 UC Berkeley Admissions}

\bigskip

Proportions of men and women \textit{applying} to the 6 departments.\\
% latex table generated in R 3.2.2 by xtable 1.8-0 package
% Fri Dec  4 01:28:36 2015
\begin{tabular}{rrrrrrrr}
  \hline
 & A & B & C & D & E & F & Overall \\ 
  \hline
Male & 31 & 21 & 12 & 15 & 7 & 14 & 100 \\ 
  Female & 6 & 1 & 32 & 20 & 21 & 19 & 100 \\ 
  diff. & 25 & 19 & -20 & -5 & -14 & -5 & 0 \\ 
   \hline
\end{tabular}\\
\bigskip

Define dummy variables \texttt{DeptA}, \ldots, \texttt{DeptF}. Comparison is \textit{balanced} for \texttt{DeptA} if M, W means of \texttt{DeptA} are similar.
% \[ \mathtt{a}_{i}=\left\{
%   \begin{array}{lr}
%     1, & $i$\, \mathrm{applied\ to\ dept\ A}\\
%     0, & \mathrm{otherwise}.
%   \end{array}
% \right.\]
% The comparison is \textit{balanced} on \texttt{a} if $\bar{\mathtt{a}}_{m}
% \approx \bar{\mathtt{a}}_{f}$, balanced on \texttt{b} if $\bar{\mathtt{b}}_{m}
% \approx \bar{\mathtt{b}}_{f}$, etc.  \pause

{\small
\begin{Schunk}
\begin{Sinput}
> xBalance(Female ~ Dept, data=UCBA, 
+          report=c("adj.means", "adj.mean.diffs"))
\end{Sinput}
\begin{Soutput}
      strata  unstrat                           
      stat   Female=0 Female=1 adj.diff         
vars                                            
DeptA        3.1e-01  5.9e-02  -2.5e-01 ***     
DeptB        2.1e-01  1.4e-02  -1.9e-01 ***     
DeptC        1.2e-01  3.2e-01  2.0e-01  ***     
DeptD        1.5e-01  2.0e-01  4.9e-02  ***     
DeptE        7.1e-02  2.1e-01  1.4e-01  ***     
DeptF        1.4e-01  1.9e-01  4.7e-02  ***     
\end{Soutput}
\end{Schunk}
}

\vfill
\end{frame}

\itnote{
\item Discuss multiple comparisons strategies
\item Hotelling's $T^{2}$; randomization test of Mahalanobis distance of imbalances from 0. 
}


\section{Additional issues in balance testing}

\subsection{Clusters and strata}
\begin{frame}{Simple and block-randomized experiments}
\framesubtitle{Example: Gerber and Green's (2000, APSR) Vote 98
  experiment}

\begin{columns}
  \begin{Column}[.3\textwidth]
    \begin{itemize}[<+->]
    \item The experiment(s)
    \item Combined diff of means check for in-person expt?
    \item (Have to 1st take sums across clusters)
    \item Combined diff of means check for phone, mail  expts?
    \item (Have to address the potential Simpson's paradox)
    \end{itemize}
  \end{Column}
  \begin{Column}[.7\textwidth]
    \igrphx{gg2k-exptdesign}
  \end{Column}
\end{columns}
  
\end{frame}

\itnote{
\item 
}



\begin{frame}{For permutation tests of block-randomized experiments}
  
  \begin{enumerate}
  \item If you're using a regression model to generate test
    statistics, \textit{consider} adding fixed effects for blocks.  I.e., fit
    the two-way ANOVA model
$$
Y = \sum_{s}\{S=s\}\alpha_{s} + Z\tau + \epsilon,\, \mathbf{E}(\epsilon) = 0.
$$
 (For randomization tests, the fixed effects aren't necessary; but
 people expect to see them, and they can be helpful.)
\item If there are strata, \texttt{RItools::xBalance()} figures
  ``adjusted mean differences'' in this way.  (Separately
  for each covariate, after summing the covariate over any designated \texttt{cluster}s.)
  \item  You \textit{have to} permute $Z$ separately within strata -- i.e.,\\
\texttt{> shuffle(z, groups=myblocks)}\\
in order to have a  valid permutation test. True whether or not $\mathrm{Pr}(Z=1|S=s)$ depends on $s$, although more variation in that probability makes things worse.
  \end{enumerate}




\end{frame}

\subsection{Covariates with NAs}

\begin{frame}[fragile]{Item missingness as data}

  \begin{itemize}
  \item (Following Rosenbaum \& Rubin, 1984, JASA.
See discussion of Rosenbaum, \textit{DOS}, ch. 9.)
  \item Example, from analysis of an observational study of coaching
    for the SAT, comparing a candidate post-stratification to a
    matching within poststrata:
  \end{itemize}

\begin{semiverbatim}
              strata racebySES          matched        
              stat    std.diff         std.diff        
vars                                                   
psat.NATRUE            0.1183  *        0.0051         
psatv                  -0.0622          0.0071         
psatm                  0.0144           0.0142         
presat.NATRUE          -0.0281          0.0289         
presatv                -0.1161 *        0.0084         
presatm                0.0275           -0.0116        
---Overall Test---
          chisquare df p.value
racebySES     47.28  8 1.4e-07
matched        0.81  8 1.0e+00
---
Signif. codes:  0 '***' 0.001 '** ' 0.01 '*  ' 0.05 '.  ' 0.1 '   ' 1     
\end{semiverbatim}

  
\end{frame}

\subsection{Covariates with funky distributions}

\note{
For now, this is a placeholder section for some slides on the combo
of Huberized loss and stratum fixed effects.  It might be filled out
with ref to some balance checks from the 2011 festschrift paper,
which contained outliers using linear splines with user-chosen
knots.
}

\begin{frame}{Test statistics for comparing means with Normal \& non-Normal data}

If treatment was assigned separately within strata, a two-way anova model is often used:
$$
Y = \sum_{s}\{S=s\}\alpha_{s} + Z\tau + \epsilon,\, \mathbf{E}(\epsilon) = 0.
$$
OLS estimates of $\tau$:
\begin{itemize}[<+->]
  \item Are optimal, both as estimates and as statistics for testing hypotheses about $\tau$, if $\epsilon \sim \mathcal{N}(0, \sigma^{2})$;
  \item Need not have the right size, and typically will have suboptimal power, otherwise.
  \item Permutation-based versions of these tests have correct size\ldots
  \item But still suffer in terms of power.
\end{itemize}

\end{frame}


\begin{frame}[fragile]{Example (Hansen 2011)}

{\small
\begin{semiverbatim}
xBalance(Coach \textasciitilde 
         psatv + pmin(psatv,40) + pmax(psatv,60) + 
         psatm + pmin(psatm,40) + pmax(psatm,60) + 
         psat.NA + \ldots) 

                    strata no.stratification          racebyses         
                     stat            adj.diff           adj.diff         
vars                                                                     
psatv                                -1.1e-01           -5.0e-01         
pmin(psatv, 40)                      9.5e-02            1.8e-01          
pmax(psatv, 60)                      -2.1e-01 .         -3.7e-01 **      
psatm                                9.1e-01  *         1.2e-01          
pmin(psatm, 40)                      2.2e-01  *         2.0e-01  *       
pmax(psatm, 60)                      1.5e-01            -5.6e-02         
psat.NATRUE                          4.6e-02  *         5.6e-02  *       
\end{semiverbatim}
$\vdots$    

}


  
\end{frame}

\section[Isolating natural experiments]{Isolating \& testing natural experiments}


\begin{frame}

  \begin{itemize}
  \item baseline comparability for Medellin data

\item pair matching for Medellin

  \end{itemize}
  
\end{frame}

\begin{frame}{Stepwise intersection-union testing (Hansen \& Sales, 2015)}
  
\end{frame}

\end{document}


\begin{frame}{Adapting Huber's M-estimation to tests w/ stratified data}


      Two approaches:
      \begin{enumerate}
      \item using Huber's loss, fit a two-way model  ---
$$
Y = \sum_{s}\{S=s\}\alpha_{s} + Z\tau + \epsilon,\, \mathbf{E}(\epsilon) = 0.
$$
The test statistic is the $\hat{\tau}$ that results.
\item using Huber's loss, fit a one-way model \textit{with no $Z$-contribution},
$$
Y = \sum_{s}\{S=s\}\alpha_{s} + \epsilon,\, \mathbf{E}(\epsilon) = 0.
$$
Write residuals from this model as $e$, $\tilde{e} = \psi(e)$.  The test statistic is the difference-of-means in $\psi(e)$ (or, equivalently, $Z'\tilde{e}$).
      \end{enumerate}



  
\end{frame}

