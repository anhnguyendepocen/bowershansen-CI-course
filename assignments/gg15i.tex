
A researcher plans to ask six subjects to donate time to an adult
literacy program. Each subject will be asked to donate either 30
($Z=0$) or 60 ($Z=1$)
minutes. The researcher is considering three methods for randomizing
the treatment. Method I is to make independent decisions for each
subject, tossing a coin each time. Method C is to
write ``30'' and ``60'' on three playing cards each, and then shuffle
the six cards. Method P tosses one coin for each of the 3 pairs
$(1,2)$, $(3,4)$, $(5,6)$, asking for 30 (60) minutes from exactly one
member of each pair. 
  
\begin{itemize}
\item[a] Discuss strengths \& weaknesses of each method.
\item[b] How would your answers to (a) change if $n: 6 \mapsto 600$?
\item[c] Determine $\EE(  Z_{1} )$  under each method.
\item[d] Determine $\EE\big(  Z_{1} + Z_{2} + \cdots + Z_{6} \big)$ under each method.
\end{itemize}
