A researcher plans to ask six subjects to donate time to an adult
literacy program. Each subject will be asked to donate either 30
($Z=0$) or 60 ($Z=1$)
minutes. The researcher is considering three methods for randomizing
the treatment. Method I is to make independent decisions for each
subject, tossing a coin each time. Method C is to
write ``30'' and ``60'' on three playing cards each, and then shuffle
the six cards. Method P tosses one coin for each of the 3 pairs
$(1,2)$, $(3,4)$, $(5,6)$, asking for 30 (60) minutes from exactly one
member of each pair. 

\begin{itemize}
\item[a] Calculate $\EE(\mathbf{Z}'\mathbf{Z})$ under each of the three methods.
\item[b] For which of the methods does $\EE
  \big[\mathbf{Z}'\mathbf{Z} -\EE (\mathbf{Z}'\mathbf{Z})\big]^{2} =
  0$?\footnote{I.e., for which does
    $\mathrm{Var}(\mathbf{Z}'\mathbf{Z}) = 0$?  (In general,
    $\mathrm{Var}(V) = \EE \big[V - \EE(V) \big]^{2} $.)}
\item[c] ``$\EE \frac{\mathbf{Z}'\mathbf{x}}{\mathbf{Z}'\mathbf{Z}}$''
  is another way of writing ``the treatment group
  \underline{\hspace{3em}} of $x$.''  Fill in the blank.
\item[d] For two of the three methods, algebraic principles we've seen
  let you can reduce
  $\EE \frac{\mathbf{Z}'\mathbf{x}}{\mathbf{Z}'\mathbf{Z}}$ to a
  familiar function of $(x_{1}, x_{2}, \ldots, x_{6}) $.  Which 2 are
  these, and why doesn't the same thing work for the third?
\end{itemize}
