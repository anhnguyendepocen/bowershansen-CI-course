%  For slides only
%%% Usage:
%% $ cat slidesonly.tex unitXX-foo.tex | pdflatex
%% $ pdfjam -o unitXX-slides{a/b/...}.pdf texput.pdf '1,{1stpage}-{lastpage}'

\documentclass{beamer} 
\newcommand{\tcw}{\textcolor{black}}
\newcommand{\mynoteonly}{}
\newcommand{\nottheirhandout}{handout:0}

% % For handout
%%% Usage:
%% $ cat handout.tex unitXX-foo.tex | pdflatex
%% $ pdfjam -o unitXX-handout{a/b/...}.pdf texput.pdf '{1stpage}-{lastpage}'
\documentclass[handout]{beamer}
\usepackage{pgfpages}
\pgfpagesuselayout{4 on 1}[letterpaper,border shrink=5mm]%\nofiles

\mode<handout>{\setbeamercolor{background canvas}{bg=black!5}}
\newcommand{\tcw}{\textcolor{structure.bg}}
\newcommand{\mynoteonly}{| handout:0}
\newcommand{\nottheirhandout}{handout:0}

%For handout + mynotes
%% Usage:
%% $ cat handout+mynotes.tex unitXX-foo.tex | pdflatex
%% $ pdfjam --nup 1x4 -o unitXX-foo-wmn{a/b/...}.pdf texput.pdf '1,{1stpage}-{lastpage}'
\documentclass[handout]{beamer}
\usepackage{pgfpages}
%%\pgfpagesuselayout{2 on 1}[letterpaper,border shrink=5mm] 
\setbeameroption{show notes on second screen=left}
\newcommand{\tcw}{\textcolor{black}}
\newcommand{\mynoteonly}{}
\newcommand{\nottheirhandout}{}

\newcommand{\igrphx}[2][width=\linewidth]{\includegraphics[#1]{images/#2}}

\renewcommand{\strut}{\rule{0pt}{3ex}}

% Frame note commands, with time budget updating
\newcounter{timeTotal}
\newcommand{\printUpdateTimeTotal}[1]{ \addtocounter{timeTotal}{#1}
  \textrm{(#1 min budgeted)} \hfill \textbf{Finish by
    \arabic{timeTotal}\ min from start}\\[1ex]}
\newcommand{\tnote}[3][10]{#2 \note{#2: #3 \\[.5ex]} \printUpdateTimeTotal{#1}}
\newcommand{\itnote}[2][10]{\note{ \begin{itemize} #2 \end{itemize}
\vfill \noindent  \printUpdateTimeTotal{#1} }}
\newcommand{\ennote}[2][10]{\note{ \begin{enumerate} #2 \end{enumerate}
\vfill \noindent  \printUpdateTimeTotal{#1} }}
\newcommand{\Note}[2][10]{\note{ #2 \mbox{ }\\ \vfill \noindent  \printUpdateTimeTotal{#1} }}

\newenvironment{Column}[1][.5\linewidth]{\begin{column}{#1}}{\end{column}}

\mode<handout>{\beamertemplatesolidbackgroundcolor{black!5}} 
\mode<article>{\usepackage{fullpage}}
\mode<presentation>
{
  \usetheme{Boadilla}
  % or ...

  \setbeamercovered{transparent}
  % or whatever (possibly just delete it)
}
\usepackage{url}
\usepackage{ulem}

\usepackage[english]{babel}
% or whatever

\usepackage[latin1]{inputenc}
% or whatever

\usepackage{times}
\usepackage[T1]{fontenc}
% Or whatever. Note that the encoding and the font should match. If T1
% does not look nice, try deleting the line with the fontenc.

\AtBeginSubsection[]
{
  \begin{frame}<beamer>
    \frametitle{Outline}
    \tableofcontents[subsectionstyle=show/shaded/hide]
  \end{frame}
}

\AtBeginSection[]
{
  \begin{frame}<beamer>
    \frametitle{Outline}
    \tableofcontents[hideothersubsections,sectionstyle=show/shaded]
  \end{frame}
}


% If you wish to uncover everything in a step-wise fashion, uncomment
% the following command: 

%\beamerdefaultoverlayspecification{<+->}

%USEFUL CODE TEMPLATES:
%\begin{itemize}[<+-| alert@+>]


% \begin{frame}[fragile]


% \begin{columns}
% \column{.4\textwidth}  
%   \begin{itemize}
%   \item<1-| alert@1> Why sample twice?
%   \item<2-| alert@2> Illustrative example.
%   \item<3-| alert@3> PSU's and SSU's.
%   \item<4-| alert@4> Stratification in combination with two-stage cluster sampling.
%   \item<5-| alert@5> Detailed example.
%   \end{itemize}
% \column{.6\textwidth}
% \only<2| handout:0>{
% \includegraphics[width=\textwidth]{nursingHomes}}
% \only<5| handout:1>{
% \includegraphics[width=\textwidth]{hosp_cover}}
% \end{columns}

% FOR INCLUDING R CODE. 
%\usepackage{listings}
%\lstset{language=R}
% \begin{frame}[fragile]
% \frametitle{Some code}
% \begin{lstlisting}
% > plot(myobj)
% > rm(myobj)
% \end{lstlisting}  
% \end{frame}

%\addtocounter{framenumber}{-1}


\usepackage{amsmath,amsthm}
\usepackage{wasysym,pifont}
\usepackage{Sweave}
\usepackage{ulem}
\usepackage{textcomp}
\usepackage{versions}
%% amsthm-type theorem environment specifications -- 
%% see amsthdoc.pdf in amscls documentation
\theoremstyle{plain}
\newtheorem{prop}{Proposition}[section]
\newtheorem{lem}[prop]{Lemma}

\newtheorem*{thm}{Proposition}
\newtheorem*{cor}{Corollary}

\theoremstyle{definition}
\newtheorem{defn}{Definition}[section]

\newcommand{\Pdistsym}{P}
\newcommand{\Pdistsymn}{P_n}
\newcommand{\Qdistsym}{Q}
\newcommand{\Qdistsymn}{Q_n}
\newcommand{\Qdistsymni}{Q_{n_i}}
\newcommand{\Qdistsymt}{Q[t]}
\newcommand{\dQdP}{\ensuremath{\frac{dQ}{dP}}}
\newcommand{\dQdPn}{\ensuremath{\frac{dQ_{n}}{dP_{n}}}}
\newcommand{\EE}{\ensuremath{\mathbf{E}}}
\newcommand{\EEp}{\ensuremath{\mathbf{E}_{P}}}
\newcommand{\EEpn}{\ensuremath{\mathbf{E}_{P_{n}}}}
\newcommand{\EEq}{\ensuremath{\mathbf{E}_{Q}}}
\newcommand{\EEqn}{\ensuremath{\mathbf{E}_{Q_{n}}}}
\newcommand{\EEqni}{\ensuremath{\mathbf{E}_{Q_{n[i]}}}}
\newcommand{\EEqt}{\ensuremath{\mathbf{E}_{Q[t]}}}
\newcommand{\PP}{\ensuremath{\mathbf{Pr}}}
\newcommand{\PPp}{\ensuremath{\mathbf{Pr}_{P}}}
\newcommand{\PPpn}{\ensuremath{\mathbf{Pr}_{P_{n}}}}
\newcommand{\PPq}{\ensuremath{\mathbf{Pr}_{Q}}}
\newcommand{\PPqn}{\ensuremath{\mathbf{Pr}_{Q_{n}}}}
\newcommand{\PPqt}{\ensuremath{\mathbf{Pr}_{Q[t]}}}
\newcommand{\var}{\ensuremath{\mathbf{V}}}
\newcommand{\varp}{\ensuremath{\mathbf{V}_{P}}}
\newcommand{\varpn}{\ensuremath{\mathbf{V}_{P_{n}}}}
\newcommand{\varq}{\ensuremath{\mathbf{V}_{Q}}}
\newcommand{\cov}{\ensuremath{\mathbf{Cov}}}
\newcommand{\covp}{\ensuremath{\mathbf{Cov}_{P}}}
\newcommand{\covpn}{\ensuremath{\mathbf{Cov}_{P_{n}}}}
\newcommand{\covq}{\ensuremath{\mathbf{Cov}_{Q}}}

\newcommand{\hatvar}{\ensuremath{\widehat{\mathrm{Var}}}}
\newcommand{\hatcov}{\ensuremath{\widehat{\mathrm{Cov}}}}

\newcommand{\sehat}{\ensuremath{\widehat{\mathrm{se}}}}

\newcommand{\combdiff}[1]{\ensuremath{\Delta_{{z}}[#1]}}
\newcommand{\Combdiff}[1]{\ensuremath{\Delta_{{Z}}[#1]}}

\newcommand{\psvec}{\ensuremath{\varphi}}
\newcommand{\psvecgc}{\ensuremath{\tilde{\varphi}}}


\newcommand{\atob}[2]{\ensuremath{#1\!\! :\!\! #2}}
\newcommand{\stratA}{\ensuremath{\mathbf{S}}}
\newcommand{\stratAnumstrat}{\ensuremath{S}}
\newcommand{\sAsi}{\ensuremath{s}}

\newcommand{\permsd}{\ensuremath{\sigma_{\Pdistsym}}}
% \newcommand{\dz}[1]{\ensuremath{d_{z}[{#1}]}}
\newcommand{\dZ}[1]{\ensuremath{d_{Z}[{#1}]}}
\newcommand{\tz}[1]{\ensuremath{t_{{z}}[#1]}}
\newcommand{\tZ}[1]{\ensuremath{t_{{Z}}[#1]}}


\newlength{\tabcolsepadj}
\setlength{\tabcolsepadj}{1.3mm}

%%% NEWBLOCK UNDEFINED BUG
\def\newblock{\hskip .11em plus .33em minus .07em}

%%% tightlist undefined control sequence bug 
%%% http://tex.stackexchange.com/questions/257418/error-tightlist-converting-md-file-into-pdf-using-pandoc
\providecommand{\tightlist}{%
  \setlength{\itemsep}{0pt}\setlength{\parskip}{0pt}}
\newcommand{\igrphx}[2][width=\linewidth]{\includegraphics[#1]{images/#2}}

\renewcommand{\strut}{\rule{0pt}{3ex}}

\newenvironment{Column}[1][.5\linewidth]{\begin{column}{#1}}{\end{column}}

\usetikzlibrary{arrows} % also see \tikzstyle spec below after \begin{doc}

\newcommand{\mlpnode}[1]{\tikz[baseline=-.5ex] \coordinate (#1) {};}
%\newcommand{\mlpnode}[1]{\raisebox{.5ex}{\pnode{#1}}}


\usepackage{xspace}

\includeversion{pedantic} % \excludeversion

\tikzstyle{every picture}+=[remember picture]


\date{ICPSR Session 2 (\today)}

\title[Day 18]{Day 18: $H^{3}$ sensitivity analysis; various}
% \author, date moved to beamer-preamble-*-all.tex


% copied from CSCAR svn repo, `dissdefs.tex`
\usepackage{xspace}
\newcommand{\satm}{\mbox{\textsc{sat-m}}\xspace}
\newcommand{\satv}{\mbox{\textsc{sat-v}}\xspace}
\newcommand{\upm}{\mbox{\textsc{urm}}\xspace}
\newcommand{\asian}{\mbox{\textsc{asian}}\xspace}
\newcommand{\presatm}{\mbox{\textsc{pre-m}}\xspace}
\newcommand{\premath}{\mbox{\textsc{pre-m}}\xspace}
\newcommand{\presatv}{\mbox{\textsc{pre-v}}\xspace}
\newcommand{\preverb}{\mbox{\textsc{pre-v}}\xspace}
\newcommand{\parentsinc}{\mbox{\textsc{incm}}\xspace}
\newcommand{\gpa}{\mbox{\textsc{gpa}}\xspace}
\newcommand{\dadsed}{\mbox{\textsc{dadsed}}\xspace}
\newcommand{\momsed}{\mbox{\textsc{momsed}}\xspace}
\newcommand{\avgeng}{\mbox{\textsc{e-gpa}}\xspace}
\newcommand{\avgmath}{\mbox{\textsc{m-gpa}}\xspace}
\newcommand{\avgnatsci}{\mbox{\textsc{ns-gpa}}\xspace}
\newcommand{\avgssci}{\mbox{\textsc{ss-gpa}}\xspace}
\newcommand{\coach}{\mbox{\textsc{coach}}\xspace}
\newcommand{\dadcoll}{\mbox{\textsc{dadcoll}}\xspace}
\newcommand{\aavg}{\mbox{\textsc{a-avg}}\xspace}

\begin{document}

\input{announcement-of-the-day}

\section{$H^{3}$ sensitivity analysis; PS $\leftrightarrow$ OV ``interaction''}


\begin{frame}
  \frametitle{Background: omitted variables}
  
  Almost all findings from observational studies could in principle be
  explained by an unmeasured variable. But how strong a confounder
  would be needed?  Rosenbaum (1987,\ldots) quantifies confounding in
  terms of propensity scores.
  
  \begin{center}
  \igrphx[height=.6\textheight]{rb2010p77}
\end{center}

  An analytically easier course uses regression to quantify confounding. 
\end{frame}

\begin{frame}
  \frametitle{Causal inference via covariance adjustment}


When we fit a model
$$
Y = a + Zb +  \mathbf{X} c + e,\,\,\, e \sim \mathcal{N}(0, \sigma^{2}),
$$
the estimated $b$ merits interpretation as the causal effect of treatment $Z$ if\\
\begin{enumerate}
\item $(Y_{t}, Y_{c}) \perp Z | X$
\item The linear relationship well-enough describes the finite population of interest.
\end{enumerate}

The first of these assumptions is usually untestable, and more central. 

\end{frame}

\begin{frame}
  \frametitle{Sensitivity analysis for the linear model}
\framesubtitle{An omitted variable}

We fit
$$
Y = a + Zb +  \mathbf{X} c + e,\,\,\, e \sim \mathcal{N}(0, \sigma^{2}),
$$
but would have liked to have fit
$$
Y = \alpha + Z\beta +  \mathbf{X} \gamma + W\zeta + e,\,\,\, e \sim \mathcal{N}(0, \sigma^{2}).
$$

\begin{enumerate}
\item<2-> How different are $b$ and $\beta$?
\item<3-> How different are the confidence intervals for $b$ and $\beta$?
\end{enumerate}

\end{frame}

 
\begin{frame}
  \frametitle{Sensitivity analysis for the linear model}
\framesubtitle{``speculation parameters''}

We can quantify the relationship of $b$ to $\beta$, and of $\mathrm{se}(b)$ to
$\mathrm{se}(\beta)$, in terms of 2 ``speculation parameters'':
\begin{enumerate}
\item $t_{w}$, the $t$-statistic associated with $W$'s coefficient in the (OLS) regression of $Z$ on $W$ and $\mathbf{X}$;
\item $R^2_{y\cdot z\mathbf{x} w}$, the
coefficient of multiple determination of the regression of $Y$ on $Z$,
$\mathbf{X}$, and $W$.
\end{enumerate}

More specifically, we can bound the upper and lower limits of conventional confidence intervals in terms of $t_{w}$ alone, or (often more sharply) in terms of $t_{w}$ and $R^2_{y\cdot z\mathbf{x} w}$.
\end{frame}
\begin{frame}
  \frametitle{$W$-insensitive confidence bounds as a function of $t_{w}$}

  \begin{prop}[Hosman, Hansen \& Holland, 2010]
Let $T>0$.  If $|t_{w}| \leq T$ then 
$$
\hat{\beta}\pm q\widehat{\mathrm{se}} (\hat{\beta}) \subseteq 
\hat{b} \pm \left( \sqrt{T^2 + q^{2} \cdot \frac{T^2 + 
n-r(\mathbf{X})-2}{n-r(\mathbf{X})-3} } \right) \widehat{\mathrm{se}}(\hat{b}).
$$
\end{prop}
\end{frame}

\frame<1-4>[label=probsstudy]
{
  \frametitle{Application: An observational study of coaching for the SAT}

Powers \& Rock (1999) sampled one in 200 SAT-I registrants in 1995-96.
\begin{itemize}
\item<2-| alert@+> The ``treatment'' is being coached for the SAT.  This
  information comes from survey responses.
\item<3-| alert@+> Outcomes, \textit{i.e.} SAT scores, come from the College Board's
administrative records.
\item<4-| alert@+> Many students took the SAT or PSAT before being coached; so
  there are pretest scores too. \pause
\item<5-| alert@+> Covariate is high-dimensional (and a little bit messy).
\end{itemize}
}

\begin{frame}  \frametitle{Sensitivity of estimates of \satm benefit of coaching }
\framesubtitle{\satm \textasciitilde \coach + \presatm + \presatv + \asian + \upm + \dadcoll + \aavg + \avgmath}

\newlength{\pholderlngth}
\settowidth{\pholderlngth}{[aims for postgrad. deg.]}
\begin{center}
\begin{tabular}{|l|l|r|} \hline
\parbox{\pholderlngth}{Excluding from predictors in regression of \coach on 
\asian,\ldots, \avgmath the
 variable}  & 
\parbox{2cm}{dullens prediction of \coach by $t=$}
    & \parbox{2.5cm}{Then if $W$ is s.t. $t_w \leq t$,
$\beta \pm 2\widehat{\mathrm{se}}(\hat{\beta})  \subseteq$} \\  
\hline
\asian &       8.1 & [-2,44] \\
\upm  &       3.2 &  [11,32] \\
\dadcoll &       7.9 & [-1,44]\\
\presatm &       2.4 &  [13,30 ]\\
\presatv  &       -2.4&  [13,30 ]  \\ 
\aavg    &      0.5 &  [16,27] \\ 
\avgmath &      -2.0& [13,29 ] \\\hline 
 \end{tabular}
\end{center}

\end{frame}

\begin{frame}
  \frametitle{Confidence limits for math effect under several omitted variable scenarios }


\begin{columns}
\column{.5\linewidth}%
{
\igrphx[width=\linewidth]{SensAnal-mathSA}

\bigskip
{\footnotesize Propensity score stratification + fixed effects regression}
}

\column{.5\linewidth}%
{
\only<2->{
\igrphx[width=\linewidth]{climits_ovsa1}

\bigskip
{\footnotesize Covariate adjustment via OLS}
}
}
\end{columns}

$H^{3}$ (2010) show this phenomenon in another example.
\end{frame}

\section[R.I. w/ modeled outcomes]{Randomization inference w/ modeled outcomes}
\subsection[U3 review]{Unit 3 review: Confidence intervals by inversion of tests}
\begin{frame}{Confidence intervals and models of effects}

  \begin{itemize}[<+->]
  \item The 3 models just considered fall under the broader model that the GOTV
generated $v$ votes per contact, some $v \in [-1, 1]$.
\item Assuming that model, the 95\% CI for $v$ is the collection of $v$s corresponding to incomplete response schedules that would not be rejected at level .05 - a confidence interval by inversion of a family of hypothesis tests.
\item This enables us to back out a confidence interval for the
  ``Complier average treatment effect,'' w/o violating the
  intention-to-treat principle! (Rosenbaum, 1997).
\item In addition, this method of confidence interval construction is
  immune to the problem of weak instruments (cf. \textit{DOS} \S~5.3 \&
  contained references).
  \end{itemize}


\end{frame}

\begin{frame}{Estimates and models of effects}
  
  \begin{itemize}
\item If you want an estimate to go with such a confidence interval,
  the convention is to report a \textit{Hodges-Lehmann} estimate ---
  the limit of 100*$(1-\alpha)$\% CIs, as $\alpha \uparrow
  1$\footnote{Assuming the limit exists and contains a single
    point. If you want to compute (rather than interpret) an HL
    estimate, get the precise definition, as e.g. in \textit{DOS} \S~2.4.3. }. 
\item There's no precise analogue of the ``standard error''\ldots
\item So you can't use $\hat{\theta} \pm z_{*}\mathrm{s.e.}(\hat
  \theta)$ as a CI. 
\item However, for the spirit of a ``standard error'', some report a 2/3 CI alongside a 95\% CI  (Mosteller \& Tukey, 1977, \textit{Data analysis and regression}).
  \end{itemize}
\end{frame}
\subsection{Example: Violence in Medell{\'i}n} 
\begin{frame}
\frametitle{Example: Violence and public infrastructure in Medell{\'i}n, Colombia}
\begin{columns}
  \begin{column}{.5\linewidth}
    \begin{itemize}
    \item 2 million residents; 16 districts
    \item Pre-intervention: 60\% poverty rate, 20\% unemployment, homicide 185 p
er 100K
    \item High residential segregation
    \item<2-> 2004-2006: infrastructure intervention for certain poor neighborho
ods.
    \end{itemize}
  \end{column}
  \begin{column}{.5\linewidth}
    \only<1>{\igrphx{medellin-conc-pov}}
\only<2-\mynoteonly>{\igrphx{medellin-gondola}}
  \end{column}
\end{columns}

\end{frame}
\itnote{
\item 2017+: Cycled back to this later in course, not during unit 3.
  Ordinarily it's too much at first pass.
\item (strata will enter the story because the end analysis involved
  matching)
}



\begin{frame}[fragile]{A model of effects for homicide rates in Medellin}

Each Medellin neighborhood provides several years of non-independent
data.  If we're willing to model Metrocable as a natural experiment,
we can borrow longitudinal data methods, without having to adopt their
dependence assumptions.   For instance, using random
 effects: 

 \begin{enumerate}[<+->]
 \item Let
   $y(T)=$ neighborhood homicide rate in year $T=2002, \ldots, 2008$.  We'll test models of
   effects of form, for rates $r \leq 0$:
$$H_{0}: y_{t}(T) = \left\{ \begin{array}{lr}\exp((T-2004) r/4)
                              y_{c}(T),& T> 2004; \\
y_{c}(T),& T\leq 2004.\end{array} \right.$$ 
 \item Given $H_{0}$, fit a random effects model of this form:
\begin{semiverbatim}
 lmer(Count \textasciitilde\  year + (year+1|nh) +\ \pause
            \alert<@+| handout:0>{offset}( nhTrt*(yr>2004)*(r/4)*(yr-2004) ),
          family=poisson, data=homd)
\end{semiverbatim}
\item Fitted params include  ``fixed'' intercept and slope,
  $\hat{\beta}_{0(r)}$ and $ \hat{\beta}_{1(r)}$,\pause  plus a
  ``random'' slope and intercept (also specific to $r$) for
  each n-hood.  
\item To test $H_{0}$, I used a test statistic comparing $t$ vs $c$
  n-hoods' fitted intercepts. For reference distribution, 
  intercepts shuffled within matched sets.
 \end{enumerate}

  
\end{frame}

\begin{frame}<\nottheirhandout>{Outcome analysis for the Medellin study}
\framesubtitle{Official statistics on n-hood violence}
  
  \begin{center}
    \igrphx[width=\linewidth]{medellin-TEs-homicide}
  \end{center}

\end{frame}



\begin{frame}<\nottheirhandout>{Outcome analysis for the Medellin
    study}
\framesubtitle{Outcomes constructed from survey responses}

\begin{center}
      \igrphx[width=\linewidth]{medellin-TEs-amenities}
\end{center}
\end{frame}

\section{matching prior to probability sampling}


\begin{frame}{Example: propensity matching prior to cluster sampling}
  \framesubtitle{to balance internal and external validity}
  \begin{itemize}[<+->]
  \item \only<2->{\sout}{Question: how did Grutter v. Bollinger affect applicants'
    responses to the diversity essay?}
\item Question: how applicants responses to diversity essay vary by
  race of applicant?
\item B/c race was so confounded w/ class, Kirkland \& Hansen (2011)
  propensity-matched, then studied a sample of propensity matched
  sets (clusters) and unmatched applicants.
  \end{itemize}
  
  \addtolength{\tabcolsep}{-\tabcolsepadj}
  {\footnotesize
  \begin{center}
\begin{tabular}{rrrrrrrrr}
  \hline & \multicolumn{2}{c}{Matched:} & \multicolumn{2}{c}{Matches:} & \multicolumn{1}{c}{Sampling} & \multicolumn{2}{c}{Unmatched:} & \multicolumn{1}{c}{Smpling} \\ & Tot. \# & Spl. \# & Tot. \# & Spl. \# & weight & Tot. \# & Spl. \# & weight. \\ 
  \hline
[0,30] & 1400 & 54 & 540 & 20 & 27 & 1600 & 10 & 160 \\ 
  (30,60] & 2600 & 45 & 880 & 15 & 59 & 9300 & 10 & 930 \\ 
  (60,100] & 610 & 47 & 200 & 15 & 14 & 2200 & 10 & 220 \\ 
   \hline
\end{tabular}
\end{center}
}
  \addtolength{\tabcolsep}{\tabcolsepadj}
\end{frame}

\end{document}
